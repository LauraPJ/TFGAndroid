% ---------------------------------------------------
%
% Trabajo de Fin de Grado. 
% Author: Laura Padrón Jorge. 
% Capítulo: Presupuesto. 
% Fichero: Cap6_BudgetEstimations.tex
%
% ----------------------------------------------------
%

\chapter{Presupuesto y puesta en marcha} \label{chap:presupuesto} 

En este capítulo se expondrán las estimaciones de recursos necesarias para poner en práctica este despliegue teniendo en cuenta el estado actual del proyecto. 

Actualmente el desglose del presupuesto de la aplicación se podría separar en dos partes: por un lado tendríamos la compra de los distintos dispositivos, y por otro lado hablaríamos del desarrollo y mantenimiento de la aplicación. Se puede contabilizar de la siguiente manera: 

\begin{itemize}
\item Para el caso de uso de los autobuses, sería necesario un beacon por parada. Si sólo se cubriesen las paradas dentro de la zona universitaria, podríamos estar hablando de unos 20 beacons distribuidos entre Anchieta y Guajara.
\item Siguiendo con el caso de uso de los eventos, se podrían calcular unas 10 zonas de eventos principales nuevamente a distribuir entre Guajara y Anchieta. Zonas como el paraninfo, aulas magnas, bibliotecas o lugares destinados específicamente para la realización de eventos.
\item Para los casos de uso que dependen de la localización, habría que realizar un análisis de cada recinto: los parkings para el módulo del aparcamiento, las aulas para el módulo de asistencia y los edificios o exteriores para el módulo de guía.
\end{itemize}

Teniendo en cuenta que cada dispositivo de Aruba lo comercializamos por 25 euros:

 
\begin{itemize}
\item Para cubrir el primer caso de uso sería necesaria una inversión de 500 euros con lo que se cubrirían 20 paradas de autobus.
\item Para las 10 zonas de eventos estaríamos hablando de unos 250 euros; sería posible añadir más beacons si fuera necesario.
\item Los casos de uso que dependen de la localización son relativos: los parkings son cerca de unos 8 y teniendo en cuenta que mínimo necesitaríamos unos 3 beacons por recinto, tendríamos un total de 600 euros. En el caso de la asistencia y guía, podríamos calcular para todos los edificios de la ULL unos 100 beacons, con lo que se cubrirían las facultades en su mayoría con un importe de 2500 euros.
\end{itemize}


Esta parte del presupuesto ascedería a unos 3.850 euros, y quedarían cubiertas los edificios, parkings y paradas de autobus cercanas al campus. La instalación y configuración de los dispositivos se desglosaría por hora calculada a 12,40 euros la hora. Se necesitarían 2 especialistas trabajando durante un mes a turno completo (8 horas) para realizar la instalación.

El total del despliegue ascendería a 3.968 euros. Todo junto contando con el precio de los dispositivos serían 7.818 euros. Los dispositivos cuentan con una garantía de un año, en caso de que el dispositivo sea considerado defectuoso se procederá al cambio de este sin coste para el cliente.


Aparte de los dispositivos habría que añadir la segunda parte del coste, que estaría formada por el mantenimiento de la aplicación y un servidor para almacenar los datos y realizar las consultas. Para esta parte se ofrecen dos opciones: 

\begin{itemize}
\item Compra del servidor dedicado más soporte y configuración servidor (cuota anual) : 1300 + 250 = 1.550 euros.
\item Configuración de un servidor ya disponible en la Universidad (sin cuota anual de soporte ni mantenimiento): 550 euros. 
\end{itemize}


Por otro lado, sería recomendable el mantenimiento de la aplicación. Ofrecemos un especialista programador en Android que se haría cargo de las incidencias que pudieran surgir con la appnen horario laboral a jornada completa, con un tiempo de respuesta de 24 horas y solución de la incidencia en 72 horas a partir de la confirmación de la misma.

El precio ascendería a 21.600 euros anuales netos durante el primer año y 18.000 euros en adelante. Este mantenimiento no incluye nuevas funcionalidades, únicamente el mantenimiento y el correcto funcionamiento de los módulos disponibles a día de hoy.

 
El presupuesto completo habría que calcularlo en función de las distintas opciones presentadas. 
