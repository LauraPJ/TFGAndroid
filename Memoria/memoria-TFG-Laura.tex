%
% ---------------------------------------------------
%
% Trabajo Fin de Grado:
% Author: Laura Padrón Jorge <gonzalezsuarezivan@gmail.com>
% Author: F. de Sande fsande@ull.es
% Fichero: main.tex
%
% ----------------------------------------------------
%
\documentclass[spanish,a4paper,14pt,oneside]{extreport}
%\documentclass[a4paper, twoside, 12pt]{book}
\usepackage[a4paper]{geometry}
\usepackage[spanish]{babel}
\usepackage[utf8]{inputenc}
%\usepackage{lscape}
\usepackage{pdflscape}
%%%%%%%%%%%%%%%%%%%%%%%%%%%%%%%%%%%%%%%%%%%%%%%%%%%%%%%%%%%%%%%%%%%%%%%%%%%%%%%%%%%%%%%%%%%%
% Next 3+3 lines select PDF or PS output (comment as apropriate)
% To switch from PDF and PS comment/uncomment here and change Makefile
\usepackage[pdftex]{color}
\usepackage[pdftex]{graphicx}
\graphicspath{{images/}}
%\usepackage[dvips]{color}
%\usepackage[dvips]{graphicx}
\usepackage{epsfig}
%\graphicspath{{images/eps/}}
%Añadidos BulletPoint
\usepackage{floatrow}
%%%%%%%%%%%%%%%%%%%%%%%%%%%%%%%%%%%%%%%%%%%%%%%%%%%%%%%%%%%%%%%%%%%%%%%%%%%%%%%%%%%%%%%%%%%%
\usepackage{algorithmic}
\usepackage[pdftex=true,colorlinks=false,urlcolor=blue,plainpages=false,pagebackref=true,citecolor=red]{hyperref} %hiperenlaces y backcites 
%%%%%%%%%%%%%%%%%%%%%%%%%%%%%%%%%%%%%%%%%%%%%%%%%%%%%%%%%%%%%%%%%%%%%%%%%%%%%%%%%%%%%%%%%%%%
% Comandos para escribir "siempre igual"
\newcommand{\BulletPoint}{\texttt{BulletPoint.{ Tecnología beacon en entornos universitarios}}}

%%% Traducimos el pseudocodigo
\renewcommand{\algorithmicwhile}{\textbf{mientras}}
\renewcommand{\algorithmicend}{\textbf{fin}}
\renewcommand{\algorithmicdo}{\textbf{hacer}}
\renewcommand{\algorithmicif}{\textbf{si}}
\renewcommand{\algorithmicthen}{\textbf{entonces}}
\renewcommand{\algorithmicrepeat}{\textbf{repetir}}
\renewcommand{\algorithmicuntil}{\textbf{hasta que}}
\renewcommand{\algorithmicelse}{\textbf{en otro caso}}
\renewcommand{\algorithmicfor}{\textbf{para}}

%%%%%%%%%%%%%%%%% Creamos un entorno para listar código fuente %%%%%%%%%%%%%%%
\newenvironment{sourcecode}
{\begin{list}{}{\setlength{\leftmargin}{1em}}\item\scriptsize\bfseries}
{\end{list}}

\newenvironment{littlesourcecode}
{\begin{list}{}{\setlength{\leftmargin}{1em}}\item\tiny\bfseries}
{\end{list}}

\newenvironment{summary}
{\par\noindent\begin{center}\textbf{Abstract}\end{center}\begin{itshape}\par\noindent}
{\end{itshape}}

\newenvironment{keywords}
{\begin{list}{}{\setlength{\leftmargin}{1em}}\item[\hskip\labelsep \bfseries Keywords:]}
{\end{list}}

\newenvironment{palabrasClave}
{\begin{list}{}{\setlength{\leftmargin}{1em}}\item[\hskip\labelsep \bfseries Palabras clave:]}
{\end{list}}


%%%%%%%%%%%%%%%%%%%%%%%%%%%%%%%%%%%%%%%%%%%%%%%%%%%%%%%%%%%%%%%%%%%%%%%%%%%%%%%
\definecolor{marron}       {rgb}{0.496, 0.203, 0.152}
\definecolor{verde-claro}  {rgb}{0.625, 0.734, 0.199}
\definecolor{oscuro}       {rgb}{0.187, 0.141, 0.285}
\definecolor{gris}     	   {rgb}{0.500, 0.500, 0.500}
\definecolor{bgd-listings} {rgb}{0.999, 0.999, 0.900}
\definecolor{gray97}{gray}{.97}
\definecolor{gray75}{gray}{.75}
\definecolor{gray45}{gray}{.45}
\definecolor{gray}{gray}{.45}
%%%%%%%%%%%%%%%%%%%%%%%%%%%%%%%%%%%%%%%%%%%%%%%%%%%%%%%%%%%%%%%%%%%%%%%%%%%%%%%%%%%%%%%%%%%%
%%% Code Listings
%\usepackage{listings} 
%\lstloadlanguages{python,C}
\definecolor{Brown}{cmyk}{0,0.81,1,0.60}
\definecolor{OliveGreen}{cmyk}{0.64,0,0.95,0.40}
\definecolor{CadetBlue}{cmyk}{0.62,0.57,0.23,0}
\definecolor{lightlightgray}{gray}{0.9}
%%%%%%%%%%%%%%%%%%%%%%%%%%%%%%%%%%%%%%%%%%%%%%%%%%%%%%%%%%%%%%%%%%%%%%%%%%%%%%%%%%%%%%%%%%%
%Evitar partir palabras al final de la línea
%\hyphenpenalty=10000
%\tolerance=1000
%%%%%%%%%%%%%%%%%%%%%%%%%%%%%%%%%%%%%%%%%%%%%%%%%%%%%%%%%%%%%%%%%%%%%%%%%%%%%%%%%%%%%%%%%%%%
% Para listados de código
\usepackage{listings}
\lstloadlanguages{C}

% Definiendo colores para los listados de código fuente - Univ. Deusto
\definecolor{violet}{rgb}{0.5,0,0.5}
\definecolor{navy}{rgb}{0,0,0.5}
\definecolor{hellgelb}{rgb}{1,1,0.8}
\definecolor{colKeys}{rgb}{0,0,1}
\definecolor{colIdentifier}{rgb}{0,0,0}
\definecolor{colComments}{rgb}{1,0,0}
\definecolor{colString}{rgb}{0,0.5,0}

%\lstset{morekeywords={pragma copy\_in copy\_out copy omp parallel private reduction shared hicuda loop\_partition over\_tblock over\_thread}}
\lstset{
        float=tbhp,
		    language = Java,
				morekeywords={llc,reduction_type,nc_result,
				              hicuda,global,alloc,shape,kernel,thread,loop_partition,tblock,over_tblock,over_thread,kernel_end,copyout,free,
											data,region,
											task,input,inout,output,
				              pragma,omp,parallel,reduction,private,shared,target,device,copy_in,copy_out,
				              acc,kernels,loop,copyin,copy,pcopy,pcopyin,collapse,gang,worker,independent},
				%\emph      ={omp,parallel,reduction,private,shared},
				emphstyle=\textbf,
        %basicstyle=\ttfamily\tiny,
        basicstyle=\ttfamily\scriptsize,
        identifierstyle=\color{colIdentifier},
        keywordstyle=\color{colKeys},
        stringstyle=\color{colString},
        commentstyle=\color[rgb]{0.133,0.545,0.133},
        columns=flexible,
        tabsize=4,
        frame=single,
        extendedchars=true,
        showspaces=false,
        showstringspaces=false,
        numbers=left,
        numberstyle=\tiny,
        breaklines=true,
        backgroundcolor=\color{lightlightgray},
        breakautoindent=true,
        captionpos=b
}

%\renewcommand{\lstlistingname}{Listing} % Los títulos de los códigos insertados se denotan con Ejemplo...   

% Otro formato más bonito para código fuente
\newcommand{\codigofuente}[3]{%
  \lstlisting[language=#1,caption={#2}]{#3}%
}
%%%%%%%%%%%%%%%%%%%%%%%%%%%%%%%%%%%%%%%%%%%%%%%%%%%%%%%%%%%%%%%%%%%%%%%%%%%%%%%
\begin{document}
\renewcommand{\lstlistingname}{Listado}% Listing -> Listado de código
%%%%%%%%%%%%%%%%%%%%%%%%%%%%%%%%%%%%%%%%%%%%%%%%%%%%%%%%%%%%%%%%%%%%%%%%%%%%%%%
% First Page
%%%%%%%%%%%%%%%%%%%%%%%%%%%%%%%%%%%%%%%%%%%%%%%%%%%%%%%%%%%%%%%%%%%%%%%%%%%%%%%

\pagestyle{empty}
\thispagestyle{empty}


\newcommand{\HRule}{\rule{\linewidth}{1mm}}
\setlength{\parindent}{0mm}
\setlength{\parskip}{0mm}

\vspace*{\stretch{0.5}}

\begin{center}
\includegraphics[scale=0.8]{images/logo_vertical}\\[10mm]
{\Huge Trabajo de Fin de Grado}
\end{center}

\HRule
\begin{flushright}
        {\Huge \BulletPoint{}} \\[2.5mm]
        {\Large Laura Padrón Jorge} \\[5mm]


\end{flushright}
\HRule
\vspace*{\stretch{2}}
\begin{center}
  \Large La Laguna, \today
\end{center}

\setlength{\parindent}{5mm}

%%%%%%%%%%%%%%%%%%%%%%%%%%%%%%%%%%%%%%%%%%%%%%%%%%%%%%%%%%%%%%%%%%%%%%%%%%%%%%%
% Signature page (add the official stamp)
%%%%%%%%%%%%%%%%%%%%%%%%%%%%%%%%%%%%%%%%%%%%%%%%%%%%%%%%%%%%%%%%%%%%%%%%%%%%%%%
\newpage
%\cleardoublepage
\thispagestyle{empty}

D. {\bf Francisco de Sande González}, con N.I.F. 42.067.050-G
profesor
Titular de Universidad
adscrito al Departamento
de Ingeniería Informática y de Sistemas
de la Universidad de La Laguna, como tutor

\bigskip

\bigskip
\bigskip
{\bf C E R T I F I C A}

\bigskip
\bigskip
\bigskip
Que la presente memoria titulada:

\bigskip
``{\it \BulletPoint{}}''

\bigskip
\bigskip
\bigskip
%Cambiar
\noindent ha sido realizada bajo su dirección por D.ª {\bf Laura Padron Jorge},
con N.I.F. 79.089.251-W

\bigskip
\bigskip

Y para que así conste, en cumplimiento de la legislación vigente y a los efectos
oportunos firman la presente en La Laguna a \today

%\cleardoublepage
\newpage
%%%%%%%%%%%%%%%%%%%%%%%%%%%%%%%%%%%%%%%%%%%%%%%%%%%%%%%%%%%%%%%%%%%%%%%%%%%%%%%
\thispagestyle{empty}

{ \flushright

\begin{LARGE}
Agradecimientos
\end{LARGE}

\hspace{3mm}

\begin{large}


\hspace{3mm}

Mis agradecimientos al profesor Francisco de Sande González por su gran labor como tutor de este proyecto, orientando este trabajo, compartiendo su conocimiento y exigiendo siempre lo mejor. 

Asimismo, me gustaría agradecer a la Universidad de La Laguna, a los Servicios TIC y a Don Juan Carlos Hernández Perdomo su colaboración en este proyecto y la puesta a disposición de los beacons, sin los cuáles habría sido imposible el desarrollo de este proyecto. 

Gracias también a Don Alberto Morales por su tiempo y dedicación en la explicación de los requisitos técnicos necesarios para la implantación de los beacons en el sistema de parking de la Ull, objeto de estudio en el desarrollo de este trabajo.

\end{large}

}

%%%%%%%%%%%%%%%%%%%%%%%%%%%%%%%%%%%%%%%%%%%%%%%%%%%%%%%%%%%%%%%%%%%%%%%%%%%%%%%%%
\newpage

\begin{huge}
Licencia
\end{huge}

\bigskip
%* Si quiere permitir que se compartan las adaptaciones de tu obra mientras se comparta de la misma manera
%y NO quieres permitir usos comerciales de tu obra indica:

\begin{center}
\includegraphics[scale=1.5]{images/by-nc-sa_88x31}\\[10mm]
{\Large \copyright~Esta obra está bajo una licencia de Creative Commons Reconocimiento-NoComercial-CompartirIgual 4.0 Internacional.
}
\end{center}


%%%%%%%%%%%%%%%%%%%%%%%%%%%%%%%%%%%%%%%%%%%%%%%%%%%%%%%%%%%%%%%%%%%%%%%%%%%%%%%
\newpage  %\cleardoublepage
\begin{abstract}
{\em

Este documento constituye el trabajo de investigación de la alumna durante el proceso de desarrollo de una aplicación para dispositivos móviles Android mediante el uso de una de las tecnologías más recientes y menos conocidas en el mercado actual: los Beacons.

\bigskip
Partimos de los conocimientos en \textit{Java} obtenidos en la asignatura: \textit{'Diseño arquitectónico y patrones'} cursada en el 
itinerario de Ingeniería del Software. Esta asignatura, impartida en el tercer curso del Grado en Ingeniería Informática de la Universidad de La Laguna, ha sido la que ha sentado los fundamentos a partir de los cuáles se ha desarrollado la aplicación.

\bigskip
Durante este proyecto, la alumna ha adquirido independencia en su trabajo, visión y planificación realizando tareas de investigación, documentación y despliegue lque han dado como resultado la obtención de conocimientos durante el proceso de trabajo.

\bigskip 
También se ha investigado la reciente tecnología de "los Beacons" que si bien aún no tiene un impacto muy grande, en unos años se espera que se empiece a utilizar con naturalidad en distintos ámbitos: turismo, comercio, enseñanza, etc.

}
\begin{palabrasClave}
Aplicaciones Android, Java, Beacons.
\end{palabrasClave}

\end{abstract}
%%%%%%%%%%%%%%%%%%%%%%%%%%%%%%%%%%%%%%%%%%%%%%%%%%%%%%%%%%%%%%%%%%%%%%%%%%%%%%%

%%%%%%%%%%%%%%%%%%%%%%%%%%%%%%%%%%%%%%%%%%%%%%%%%%%%%%%%%%%%%%%%%%%%%%%%%%%%%%%
\newpage  %\cleardoublepage
\begin{abstract}
{\em

The aim of the Project has been the development of an application for Android devices which will be using beacon technology for some of its main features.

\bigskip
Based on the knowledge of \textit{Java} obtained in the subject: \textit{'Architectural Design and Patterns'} taken in the
Software Engineering Branch (Given in third year of Computing Engineering Degree from \textit{'La Universidad de La Laguna'}).In this work we have acquired the basic knowledge needed to develop Android applications introducing us in the development of applications related to Beacon Technology.

\bigskip
Moreover, the student has learned independence in her work and gained vision and planification aptitudes, developing different labors of investigation, documentation, development and deployment, that have come to give her vast knowledge during the development of this project.

\bigskip
Apart from all this, it's of great value for the student to get to know this new Beacon technology. I have investigated and learned from this new technology, which at the moment is not well known, but in the close future it is expected to get more attention in different sectors, such as tourism, trading or learning.
}

\begin{keywords}
Application for Android, Java, Beacons.
\end{keywords}

\end{abstract}
%%%%%%%%%%%%%%%%%%%%%%%%%%%%%%%%%%%%%%%%%%%%%%%%%%%%%%%%%%%%%%%%%%%%%%%%%%%%%%%

%%%%%%%%%%%%%%%%%%%%%%%%%%%%%%%%%%%%%%%%%%%%%%%%%%%%%%%%%%%%%%%%%%%%%%%%%%%%%%%
\newpage{\pagestyle{empty}}
\thispagestyle{empty}

%%%%%%%%%%%%%%%%%%%%%%%%%%%%%%%%%%%%%%%%%%%%%%%%%%%%%%%%%%%%%%%%%%%%%%%%%%%%%%%


\pagestyle{myheadings} %my head defined by markboth or markright
% No funciona bien \markboth sin "twoside" en \documentclass, pero al
% ponerlo se dan un montón de errores de underfull \vbox, con lo que no se
% ha puesto.
\markboth{Laura Padrón Jorge}{BulletPoint}

%%%%%%%%%%%%%%%%%%%%%%%%%%%%%%%%%%%%%%%%%%%%%%%%%%%%%%%%%%%%%%%%%%%%%%%%%%%%%%%
%Numeracion en romanos
\renewcommand{\thepage}{\roman{page}}
\setcounter{page}{1}

%%%%%%%%%%%%%%%%%%%%%%%%%%%%%%%%%%%%%%%%%%%%%%%%%%%%%%%%%%%%%%%%%%%%%%%%%%%%%%%

\tableofcontents

%%%%%%%%%%%%%%%%%%%%%%%%%%%%%%%%%%%%%%%%%%%%%%%%%%%%%%%%%%%%%%%%%%%%%%%%%%%%%%%
\newpage{\pagestyle{empty}}

\listoffigures

%%%%%%%%%%%%%%%%%%%%%%%%%%%%%%%%%%%%%%%%%%%%%%%%%%%%%%%%%%%%%%%%%%%%%%%%%%%%%%%
\newpage{\pagestyle{empty}}

%\listoftables

%%%%%%%%%%%%%%%%%%%%%%%%%%%%%%%%%%%%%%%%%%%%%%%%%%%%%%%%%%%%%%%%%%%%%%%%%%%%%%%
\newpage{\pagestyle{empty}}

%%%%%%%%%%%%%%%%%%%%%%%%%%%%%%%%%%%%%%%%%%%%%%%%%%%%%%%%%%%%%%%%%%%%%%%%%%%%%%%
%Numeracion a partir del capitulo I
\renewcommand{\thepage}{\arabic{page}}
\setcounter{page}{1}


% ==========================================================
% --------               Capítulos                ----------
% --------    Estan en el directorio capitulos/   ----------
% ==========================================================
% ---------------------------------------------------
%
% Proyecto de Final de Carrera: 
% Author: Laura Padron Jorge <alu0100703511@ul.edu.es>
% Introducción
% Fichero: Prologo.tex
%
% ----------------------------------------------------

\chapter*{Introducción}
\addcontentsline{toc}{chapter}{Introducción} 

Este documento comprende el trabajo de investigación y desarrollo realizado por el autor en la consecución de su Trabajo de Fin de Grado (TFG), con el que pondrá fin a sus estudios del Grado en Ingeniería Informática cursados en la Escuela Técnica Superior de Ingeniería Informática (ETSII) de la Universidad de la Laguna (ULL).

%
% ---------------------------------------------------
%
% Proyecto de Final de Carrera:
% Author: Laura Padrón Jorge <alu0100703511@ull.edu.es>
% Capítulo: Objetivos 
% Fichero: Cap1_Goals.tex
%
% ----------------------------------------------------
%


\chapter{Objetivos} \label{chap:Objetivos}  

Este TFG tiene los siguientes objetivos principales:

	
\begin{itemize}
\item  	Por un lado se pretende ampliar los conocimientos en tecnologías móviles en el sistema operativo \textit{Android} \cite{URL::Android} y en el desarrollo de aplicaciones para este sistema operativo.
\item Por otro lado, también se pretende que la alumna se familiarice con el uso de herramientas de control de versiones utilizando GitHub \textit{Github} \cite{URL::Github} y de edición de textos técnicos utilizando \textit{LaTeX}  \cite{URL::LaTeX}.
\item Otro objetivo presente en este TFG es que la alumna investigue y profundice en una tecnología reciente en el mercado, los \textit{Beacons} \cite{URL::Beacon}.
\item  Por último, se espera que la estudiante obtenga independencia en su trabajo, visión y planificación poniendo en marcha tareas de investigación, documentación, desarrollo y despliegue utilizando tanto los conocimientos adquiridos durante la carrera, como aquellos que se irán aprendiendo durante el progreso de este trabajo.
\end{itemize}
% ---------------------------------------------------
%
% Trabajo de Fin de Grado. 
% Author: Laura Padrón Jorge. 
% Capítulo: Tecnologías utilizadas en el Trabajo de Fin de Grado. 
% Fichero: Cap2_Technology.tex
%
% ----------------------------------------------------
%

\cleardoublepage
\chapter{Herramientas y Tecnologías utilizadas} \label{chap:Tecnologias} 

Este capítulo tiene como objetivo presentar las distintas herramientas software y tecnologías empleadas por la alumna en el desarrollo de su TFG.

\section{Herramientas de Desarrollo}

A continuación se explicarán brevemente las distintas herramientas software utilizadas en el proyecto. 

\subsection{Android Studio}

Android Studio \cite{URL::AndroidStudio} es el IDE (Entorno de Desarrollo Integrado) oficial para el desarrollo de aplicaciones en Android, basado en IntelliJ IDEA \cite{URL::IntelliJIDEA}. Android Studio ofrece una serie de funcionalidades que han facilitado a la desarrolladora numerosas tareas, entre ellas podemos destacar:


\begin{itemize}
\item Un sistema de compilación basado en Gradle\cite{URL::Gradle} que ha simplificado tanto la inserción de dependencias de las distintas librerías que se han tenido que utilizar, como la compilación de la aplicación.
\item Un emulador rápido y fácil de utilizar que ha ayudado a visualizar las distintas pantallas durante el desarollo aunque no ha sido de mucha utilidad para probar el funcionamiento al ser dependiente la app de la tecnología Bluetooth.
\item La facilidad para publicar cambios a aplicaciones ya funcionando sin tener que eliminar y volver a crear un nuevo APK parando la app.
\item Un sistema de visualización de las diferentes pantallas muy completo, con soporte visual para añadir componentes y cambiar atributos fácilmente.
\item Un sistema de depuración, con una interfaz sencilla e intuitiva.
\end{itemize} 

\begin{figure}[h]
	\centering
	\includegraphics[width=\columnwidth]{androidstudio}
	\caption{Android studio, un IDE flexible e intuitivo.}
	\label{fig:androidstudio}
\end{figure}

Se ha utilizado este IDE frente a otros como Eclipse + ADT \cite{URL::eclipseADT} debido a que en la actualidad es el IDE oficial con soporte de Google. Se ha preferido aprender a utilizar este entorno con vistas al futuro, ya que parece que se consolidará como el preferido para los desarrolladores Android.

\subsection{LaTex}

LaTeX \cite{URL::LaTeX} es un sistema de composición de textos, orientado a la creación de documentos que presenten una alta calidad tipográfica. Por sus características y posibilidades, es usado especialmente en la generación de artículos y libros científicos que incluyen, entre otros elementos, expresiones matemáticas, gráficos o figuras.


LaTeX está formado por un gran conjunto de macros de TeX, escrito por Leslie Lamport en 1984, con la intención de facilitar el uso del lenguaje de composición tipográfica, creado por Donald Knuth. LaTeX es software libre bajo licencia LPPL.


Se ha decidido utilizar este sistema debido al carácter profesional que le aporta a los documentos. Ha sido una buena oportunidad para aprender a usar un sistema de composición de texto como este, ya que en un futuro puede ser beneficioso el saber manejar esta herramienta. 


Si bien es cierto, que el uso de esta herramienta frente a otros editores más familiares ha sido algo tedioso en el inicio, es verdad que una vez acostumbrada a utilizarla ha resultado ser múy eficaz. En el proceso de aprendizaje se recurrió principalmente a manuales por internet, alguno a destacar en español sería:  \cite{URL::manualLatex}

\subsection{Github}

GitHub es una forja (plataforma de desarrollo colaborativo) para alojar proyectos utilizando el sistema de control de versiones Git. Utiliza el framework Ruby on Rails por GitHub, Inc. (anteriormente conocida como Logical Awesome). Desde enero de 2010, GitHub opera bajo el nombre de GitHub, Inc. El código se almacena de forma pública, aunque también se puede hacer de forma privada, creando una cuenta de pago.


Se ha decidido crear un repositorio en esta plataforma para poder llevar un control y una trazabilidad del proyecto. El tutor y la alumna han trabajado en este repositorio de manera conjunta. En el caso del tutor, principalmente para revisar el seguimiento semanal y llevar un control de las tareas. En el caso de la alumna, para tener un repositorio donde subir los distintos elementos que se han ido generando a lo largo del trabajo. Aparte de este repositorio, también se ha abierto un segundo repositorio asociado a la OSL para subir el código una vez terminado como parte del Programa de Apoyo a Trabajos Finales Libres (PATFL) \cite{URL::PATFL} de la ULL. Accesible libremente desde: \cite{URL::repositorioAplicacion}


Mediante el uso de este repositorio, la alumna ha conseguido ampliar sus conocimientos en Git y familiarizarse con la interfaz de GitHub. Previamente se había utilizado como repositorios GitLab, SVN y RTC en otros proyectos, por lo que no ha sido una complicación mayor utilizar este sistema.


\begin{figure}[h]
	\centering
	\includegraphics[width=\columnwidth]{github}
	\caption{La plataforma de desarrollo colaborativo GitHub.}
	\label{fig:github}
\end{figure}

\section{Tecnologías utilizadas}

A continuación se explicarán las distintas tecnologías utilizadas en el desarrollo de la aplicación.

\subsection{El Sistema Operativo Android}

Android es un sistema operativo que emplea Linux en la interfaz del hardware.  Los componentes del S.O subyacentes se codifican en C o C++ pero las aplicaciones se desarrollan en Java. De esta manera Android asegura una amplia operatividad en una gran variedad de dispositivos debido a dos hechos: la interfaz en Linux ofrece gran potencia y funcionalidad para aprovechar el hardware, mientras que el desarrollo de las aplicaciones en Java permite que Android sea accesible para un gran número de programadores conocedores del código.

Este S.O fue diseñado principalmente para dispositivos móviles con pantalla táctil: smartphones, tablets y otros dispositivos como televisores o automóviles. Fue desarrollado inicialmente por Android Inc., empresa que fue respaldada económicamente por Google y más tarde comprada por esta misma empresa.

Actualmente tiene una gran comunidad de desarrolladores creando aplicaciones para extender la funcionalidad de los dispositivos. A fecha de hoy existen más de un millón de aplicaciones disponibles para la tienda oficial de Apps de Android: Google Play \cite{URL::GooglePlay} sin tener en cuenta las aplicaciones de otras tiendas no oficiales, como por ejemplo, la tienda de aplicaciones de Samsung Apps \cite{URL::SamsungApps}.

\subsection{Los Beacons}

Los \textit{''Beacons''} \cite{URL::Beacon}, cuya traducción del inglés equivaldría a \textit{'balizas'} o \textit{'faros'}, son una tecnología emergente que desde algunos años se intenta abrir paso en el mercado. Como su propio nombre indica, estos dispositivos intentan ser un mecanismo de guía, dando una solución al posicionamiento en interiores, donde otras tecnologías, como el GPS o el Wifi dejan de funcionar o resultan imprecisas. Sin embargo, estos no son los únicos usos de los beacons, actualmente muchas empresas están ampliando sus usos a otros campos.


A continuación se intentará responder a las preguntas más frecuentes que nos pueden surgir con respecto a esta tecnología:


\begin{itemize}
\item ¿Qué es un Beacon?
\item ¿Cómo funcionan estos dispositivos?
\item ¿Qué rango de alcance poseen?
\item ¿Con qué dispositivos móviles son compatibles? 
\item ¿Qué ventajas y desventajas tienen con respecto a otras tecnologías?
\item ¿Qué usos se le ha dado a esta tecnología hasta ahora?
\end{itemize}

\begin{figure}[!h]
        \begin{floatrow}
        \ffigbox{\includegraphics[width=\textwidth/2]{estimoteBeacon}}{\caption{Uno de los beacons de la compañia Estimote}\label{fig:estimote_Beacon}}
        \ffigbox{\includegraphics[width=\textwidth/2]{estimoteSticker}}{\caption{Uno de los beacons de la compañia Estimote en formato Pegatina}\label{fig:estimote_Beacon_Sticker}}
        \end{floatrow}
\end{figure}

\subsubsection{¿Qué es un Beacon?}

Para los que no hayan oido este término, en el marco en el que nos movemos, hace referencia a un pequeño dispositivo (sus tamaños varían de uno a otro, pero siempre de tamaño reducido) que emite señales de onda corta utilizando la tecnología Bluetooth \cite{URL::Bluetooth}. Estas señales contienen una pequeña cantidad de información que es recibida por dispositivos móviles con tecnología Bluetooth dentro de un rango de cobertura variable dependiendo del beacon y su configuración. Normalmente, la fuerza de esta señal y su frecuencia son configurables.

\begin{figure}[h]
	\centering
	\includegraphics[width=\columnwidth]{beaconPhone}
	\caption{Representación de un beacon emitiendo mediante Bluetooth a un dispositivo móvil}
	\label{fig:beaconBluetooth}
\end{figure}

El funcionamiento de un beacon es sencillo: El beacon emite una señal ininterrumpida que es captada por los dispositivos móviles dentro de su radio de cobertura. La señal nos ofrece información que nos sirve para detectar y localizar estos dispositivos. Esta señal es captada por una aplicación previamente instalada en un dispositivo móvil que esté programada para recibirla y reaccionar de manera acorde a la información recibida.


Hay que tener en cuenta que esta señal es unilateral: los beacons son capaces de enviar pero no están preparados para recibir. También hay que tener en cuenta, que la mayoría de los beacon actuales en el mercado transmiten información preconfigurada, confiando en la aplicacion móvil para utilizar la información; sin embargo es muy posible que esto cambie en un futuro, ampliando las posibilidades de los beacons.

\subsubsection{¿Como funcionan estos dispositivos?}

Los beacons usan Bluetooth Low Energy (BLE) \cite{URL::BluetoothLowEnergy}, una versión del protocolo Bluetooth diseñada para usar mucha menos energía y enviar menos información. Los beacons funcionan con baterías cuyo tiempo de vida depende de la configuración establecida, teniendo en cuenta la emisión de la señal (fuerza y frencuencia) y tiempo de hibernación. Sus tiempos de vida son variables, pudiendo durar desde un mes hasta varios años. 

\begin{figure}[h]
	\centering
	\includegraphics[width=\columnwidth]{estimoteBeaconInside}
	\caption{Interior de un beacon de Estimote}
	\label{fig:beaconInside}
\end{figure}

Independientemente de lo que se pueda pensar, los beacons en sí mismas no transmiten información significativa, transmiten identificadores cortos junto con información customizable breve, que son interpretadas por una aplicación que sabe lo que tiene que hacer con esa información y que es la que se encarga de procesar la información y realizar una acción pertinente.

Este identificador se divide en tres partes: 

\begin{itemize}
\item \textit{''UUID''} \cite{URL::UUID} : corresponde con una ID dada por el vendedor e identifica el beacon en cuestión.
\item ID Superior : customizables y utilizadas con un significado específico que puede identificar una acción o parámetro. 
\item ID Inferior: customizables y utilizadas al igual que la superior con un significado específico que se puede usar para identificar una acción o parámetro.
\end{itemize}

\begin{figure}[h]
	\centering
	\includegraphics[width=\columnwidth]{identity}
	\caption{Numeros identificativos de los beacons}
	\label{fig:beaconId}
\end{figure}

\subsubsection{¿Qué rango de alcance poseen?}

Actualmente los beacons en el mercado presentan un rango de aproximadamente 70 metros sin obstáculos. Está demostrado que este rango disminuye significativamente al atravesar paredes de metal o ladrillo, otros materiales disminuyen en menor medida el rango. 

Las aplicaciones que trabajan con beacons suelen definir acciones en tres rangos de distancia principalmente: 

\begin{itemize}
\item Lejos: diseñado para que el dispositivo móvil pueda lanzar una acción cuando estás en el rango exterior de un beacon, es decir, cuando se entra en el rango del beacon.
\item Cerca: diseñado para que el dispositivo móvil pueda lanzar una acción cuando estás en el rango interior del beacon. 
\item Inmediato: Diseñado para que el dispositivo móvil pueda lanzar una acción cuando te encuentras muy cercano al beacon.
\end{itemize}

\begin{figure}[h]
	\centering
	\includegraphics[width=\columnwidth]{BeaconsRange}
	\caption{Ejemplificación del rango de un beacon}
	\label{fig:beaconRange}
\end{figure}

Sin embargo, esto depende de como esté diseñada la aplicación, es posible lanzar acciones a una distancia determinada sin tener en cuenta estos rangos mencionados anteriormente, ya que en todo momento es posible conocer la distancia a la que nos encontramos del beacon.

\subsubsection{¿Con qué dispositivos funcionan?}

Las beacons son compatibles con todos los dispositivos que soporten Bluetooth Low Energy, pero para que las señales de los beacons sean detectadas por tu dispositivo, se ha de tener activado el Bluetooth. 


En dispositivos con IOS7 \cite{URL::IOS7} o superior, el dispositivo puede estar constantemente buscando dispositivos BLE y despertar a las aplicaciones implicadas cuando entran en el rango de los beacons, incluso estando cerradas las aplicaciones.


En dispositivos Android \cite{URL::Android} el sistema operativo no está preparado para escanear dispositivos BLE, por lo que son las aplicaciones las que se tienen que encargar de escanear las proximidades buscando beacons, esto supone que las aplicaciones tienen que estar funcionando, despiertas (aunque sea en segundo plano). Existen librerías que solucionan esta limitación, haciendo que la aplicación escanee cada cierto tiempo incluso en segundo plano o estando dormida, pero no es muy eficaz y suele tener incompatibilidades ya que induce conflictos con el Sistema Operativo. Un ejemplo de estas incompatibilidades lo podemos ver en este hilo de discusión \cite{URL::Incompatibilidades} de los desarrolladores de la librería AltBeacon, librería que se ha usado para BulletPoint y que mencionaremos más tarde.

Por último en dispositivos Windows Phone \cite{URL:WindowsPhone} o Blackberry \cite{URL:Blackberry} existen diferentes niveles de compatibilidad pero en los que soportan BLE, su funcionamiento es similar al de los dispositivos Android, por lo que no nos pararemos a analizarlo. 

\subsubsection{¿Qué ventajas y desventajas tienen con respecto a otras tecnologías?}

A la hora de hablar de los beacons existen una serie de ventajas pero también podemos encontrar algunas desventajas que iremos detallando a continuación. 

Las principales ventajas que se distinguen a la hora de hablar de los beacons son las siguientes: 

\begin{itemize}
\item A diferencia de la tecnología GPS, la activación del bluetooth consume mucho menos batería. 
\item Es una tecnología que puede ser dependiente de la red de datos. 
\item A diferencia de la tecnología GPS, sigue funcionando con precisión en el interior de los edificios.
\end{itemize}

En cuanto a las desventajes que nos podemos encontrar destacamos:

\begin{itemize}
\item Dependen de aplicaciones instaladas en el dispositivo móvil para funcionar. 
\item Es necesario tener el bluetooth activado, lo que consume batería durante el tiempo que esté activado. 
\item Su utilidad depende de la voluntad de terceros de utilizar estos dispositivos, configurarlos y distribuir las aplicaciones.
\end{itemize}

\subsubsection{¿Qué usos se le ha dado a esta tecnología hasta ahora?}

Por ahora esta tecnología se ha utilizado en ambiente muy diversos y con distintas funcionalidades. De los más conocidos podríamos destacar los siguientes: 

\vspace{5mm}

\textsl{\textbf{{Clevedon School (K-12)}}}

\vspace{2mm}

Este ejemplo es bastante significativo ya que se aplicó en el mismo entorno en el que queremos trabajar, una institución de enseñanza universitaria. Después de desplegar cerca de 1200 iPads  la universidad de Clevedon utilizó esta tecnología junto con su aplicación universitaria ya existente. 

\begin{figure}[H]
	\centering
	\includegraphics[width=\columnwidth]{ClevedonApp}
	\caption{La aplicación de Clevedon School}
	\label{fig:beaconRange}
\end{figure}


Han sido capaces de crear una manera fácil para que los profesores puedan añadir recursos que se envian automáticamente a los alumnos transitando diferentes zonas en diferentes horarios.Para realizar este trabajo de manera eficiente fue necesario la creación de una Interface para la gestión de los recursos en las diferentes Beacons.


Esta interfaz junto con la aplicación móvil es capaz de: 

\begin{itemize}
\item Programar los recursos para distribuirse a una hora del día especificada. 
\item Programar el material para ser distribuido en un momento determinado durante una clase o evento. 
\item Poner los recursos a disposición de los alumos que se encuentren en una localización específica.
\end{itemize}

Utilizando estos tres recursos, la aplicación, la interfaz y los beacons han sido capaces de crear un entorno interactivo y eficiente motivando tanto a profesores como estudiantes. 

\vspace{5mm}

\textsl{\textbf{{Cleveland Cavaliers Stadium y Levi's Stadium}}}

\vspace{2mm}

Dos de los ejemplos más conocidos han sido los despliegues que se han realizado en estos dos estadios. 

\begin{figure}[H]
	\centering
	\includegraphics[width=\columnwidth]{LevisStadium}
	\caption{La aplicación de Levi's Stadium}
	\label{fig:levisStadium}
\end{figure}

Por un lado tenemos el despliegue del estadio de Levi's , cuya intención ha sido la de ayudar a sus fans a navegar por el estadio dadas sus dimensiones.En este caso los beacons (de la comañía Aruba Networks) se utilizan en conjunto con puntos de acceso y repetidores situados por toda la infraestructura de manera que queda el estadio cubierto. Con la aplicación los fans también son capaces de ver repeticiones de las jugadas y pedir comida directamente desde sus dispositivos móviles.


Un punto importante de este  despliegue ha sido la monitorización continua del funcionamiento de los beacons, incluyendo si estan en funcionamiento o necesitan bateria nueva. Los beacons son también más económicos que los puntos de acceso WiFi, lo cual les ha beneficiado.

En el caso del estadio de Cleveland, los beacons se encargan de proveer al usuario de información personalizada dependiendo del lugar y la hora. En algunos casos videos, ofertas promocionales y contenido adicional.

\vspace{5mm}

\textsl{\textbf{{Orlando Int'l Airport}}}

\vspace{2mm}

Otro despliegue exitoso de esta tecnología ha sido en el Aeropuerto Internacional de Orlando , donde mediante el uso de los beacons y de una aplicación móvil propia han sido capaces de proporcionar una serie de funcionalidades de vital importancia en una infraestructura como el aeropuerto: 

\begin{itemize}
\item Navegación paso por paso a través de cerca de 1000 establecimientos o servicios dentro del aeropuerto. 
\item Actualizaciones inmediatas de la información de los vuelos. 
\item Intrucciones a puntos de interés criticos como puntos de recogida de equipaje, puertas de embarque o puestos de información.
\end{itemize}

\begin{figure}[H]
	\centering
	\includegraphics[width=\columnwidth]{orlandoAirport}
	\caption{La aplicación del Aeropuerto Internacional de Orlando}
	\label{fig:orlandoAirport}
\end{figure}

El siguiente punto sería ampliar la opción a los establecimientos de ofrecer anuncios o promociones, opción que mantienen abierta y no se descarta en un futuro.


Esta información ha sido extraída de: \cite{URL::Articulo} 

\subsection{CouchBase Server}

Couchbase Server es una base de datos NoSQL con una arquitectura distribuida orientada al rendimiento, escalabilidad y disponibilidad. Da la oportunidad a los desarrolladores de construir aplicaciones de manera sencilla y rápida combinando la flexibilidad del JSON y la tecnología NoSQL.

\subsubsection{¿Por qué utilizar Couchbase Server?}

Hemos decidido utilizar esta tecnología por su flexibilidad y potencia para almacenar documentos fácilmente. Además resulta muy sencillo integrarla con la tecnología móvil mediante el uso de una base de datos reducida dentro del dispositivo móvil y sincronizándola con la base de datos principal del servidor mediante una puerta de sincronización.

\begin{figure}[H]
	\centering
	\includegraphics[width=\columnwidth]{couchbaseexplanation}
	\caption{Sincronización de CouchBase Server con CouchBase Lite mediante Sync Gateway}
	\label{fig:couchbaseexplanation}
\end{figure}

En este caso se ha configurado el servidor en un ordenador portátil haciendo uso de las indicaciones de la página web del producto. Se procederá a desglosar brevemente los pasos seguidos a la hora de configurar el servidor.


\subsubsection{Configuración de la arquitectura}

Para configurar el servidor se han seguido los siguientes pasos: 


\begin{itemize}
\item Descargar la versión Community del producto desde la página web siguiendo el enlace: \cite{URL::couchbaseDownload}. 
\item Seguir los pasos de la página de desarrolladores para la instalación y configuración: \cite{URL::couchBaseGuide}. 
\item Una vez configurado CouchBase Server procederemos a descargar Sync Gateway que será el servicio que se encarge de sincronizar el contenido del nuestra aplicación al servidor, para ello lo descargaremos de la página al igual que el servidor siguiendo el link: \cite{URL::couchbaseDownload} .
\item Para vincular el servidor con Sync Gateway es necesario hacer uso de un fichero de configuración con el que lanzaremos el servicio Sync Gateway.
\item Una vez configurado Sync Gateway, ya tenemos el canal de configuración entre el servidor y la aplicación, para utilizar la base de datos móvil seguiremos los pasos desglosados en: \cite{URL::couchBaseLite} .
\end{itemize}


En este caso se ha tenido que conectar el dispositivo al ordenador. Tanto el ordenador como el dispositivo móvil han de estar en la misma red y hemos utilizado la dirección IP de la máquina para realizar las peticiones del móvil al servidor (alojado en el portátil) al utilizar la API. De esta manera se ha comprobado el funcionamiento del servidor, del servicio de sincronización y de la base de datos versión Lite en el dispositivo móvil.


\subsection{La librería AltBeacon}


\subsubsection{¿Qué es AltBeacon?}

Se puede definir AltBeacon como una especificación que: 

\begin{itemize}
\item Define el formato del mensaje publicitario que los beacons transmiten a través de BLE (Bluetooth Low Energy).
\item Intenta crear un mercado abierto y competitivo para implementaciones usando proximidad con los beacons.
\item Puede ser utilizada por todos gratuitamente, sin cuotas ni compromisos.
\item No favorece a ningún proveedor sobre otro. Las limitaciones vienen determinadas por los estándares técnicos del proveedor.
\end{itemize}

A continuación se profundizará en el funcionamiento y configuración de la librería AltBeacon, librería que cumple con la especificación AltBeacon y que se ha utilizado para trabajar en Android.

\subsubsection{Configuración}

Para trabajar con esta librería en Android Studio solo hemos tenido que importarla mediante el uso de Gradle a nuestro proyecto como se explica en : \cite{URL::importGradle}


También es posible descargarla desde la página oficial, donde además podremos econtrar diferentes versiones de la misma, para ello podemos hacer uso del siguiente enlace y seleccionar la versión deseada:  \cite{URL::versionAltBeacon}


\subsubsection{Funcionamiento}

Prácticamente la funcionalidad de esta librería se centra en dos elementos principales: 

\begin{itemize}
\item \textit{''Monitoring''} que sería algo así como supervisar, saber que beacons se encuentran en una región o si ha entrado o salido un beacon de una región.
\item \textit{''Ranging''} que hace referencia a rastrear, permitiendo saber a que distancia se encuentran los beacons en todo momento dentro de una región.
\end{itemize}

Utilizando estas dos funcionalidades la aplicación es capaz de controlar, monitoreando y rastreando los distintos beacons en una determinada región. En la página web de la librería se pueden encontrar ejemplos básicos\cite{URL::altbeaconSamples} de como se realizan estas dos funciones en la sección \textit{''Samples''}, además en la sección \textit{''Documentación''} se pueden encontrar también algunos artículos, que pueden resultar interesantes dependiendo del tipo de aplicación que estes desarrollando.





%
% ---------------------------------------------------
%
% Proyecto de Final de Carrera:
% Author: Laura Padrón Jorge <alu0100703511@ull.edu.es>
% Capítulo: Objetivos 
% Fichero: Cap1_Goals.tex
%
% ----------------------------------------------------
%


\chapter{Beacons en entornos universitarios} \label{chap:BeaconsEntornosUniversitarios}  


En este capítulo realizaremos un análisis de los posibles casos de uso de la tecnología beacon en el ámbito universitario.

 
\section{Aplicaciones móviles en entornos universitarios}


Actualmente las posibilidades de las aplicaciones móviles para entornos universitarios se presentan amplias, cada universidad intenta tener su propia aplicación siguiendo un patrón similar. Realizando una investigación general de las aplicaciones disponibles en el mercado observamos que estas aplicaciones se centran en ofrecer servicios propios (servicio de correo, moodle, chat entre usuarios,etc), mantener al alumno informado y agilizarle los trámites mayoritariamente. 

En un principio, estas aplicaciones se enfocaban a atraer alumnos, centrandose en la calidad de la universidad y mostrando las posibilidades que ofrecían, sin embargo, con el paso de los años y el desarrollo creciente de las aplicaciones móviles, se muestra un cambio en esta estrategia. Ahora las aplicaciones tienen una doble función, no sólo buscan el acceso de nuevos estudiantes sino que también intentan mejorar la experiencia de los alumnos existentes y acercar a los nuevos a la experiencia de la universidad. 



Algunos ejemplos posibles los encontramos en el marketplace de google:

\begin{itemize}
\item Universidad Galileo: \cite{URL::galileo}
\item Univerisidad de Valladolid: \cite{URL::valladolid}
\item Universidad de Oviedo: \cite{URL::oviedo}
\end{itemize}


Uno de los hechos que podemos observar es que las universidades están intentando obtener una solución rápida para desarrollar su app, una de estas soluciones es la creación de plantillas web optimizadas de su sitio web, lo que podemos considerar una opción rápida con un coste bajo.


Hoy en día casi todos los estudiantes tienen acceso a un dispositivo móvil y cuentan con una tarifa de internet. En un futuro próximo con la aparición de estos dispositivos ya nos surge la pregunta ¿Serán capaces estos dispositivos de transformar la educación? y en caso aformativo ¿De que manera? 

\section {Posibles casos de uso de la tecnología beacon en entornos universitarios}

Como mencionaba antes, una de las posibilidades que se presentan para explotar esta tecnología se encuentra en las instituciones de enseñanza, las cuales podrían utilizar los beacons para facilitar a sus alumnos, profesores y demás personal involucrado  una serie de servicios de gran utilidad.


Sin embargo, para utilizar esta tecnología es necesario cumplir una serie de condiciones:

\begin{itemize}
\item Tener instalada la aplicación en su dispositivo móvil.
\item Tener activado el bluetooth.
\item La aplicación ha de estar "despierta".
\item Los beacons han de estar desplegados y configurados correctamente en lugares clave donde el rango sea óptimo.
\end{itemize}

En el caso de dispositivos Apple no es necesario tener activado el bluetooth ya que el SO se encarga de captar las señales BLE, aparte tampoco es necesario que la app esté despierta ya que nuevamente el SO se encarga de despertar a la aplicación involucrada. Sin embargo Apple no ha desarrollado un IBeacon físico aún, aunque en un futuro, se espera que sus dispositivos móviles puedan funcionar como un beacon bidirecional. Cabe destacar que existen librerías que se encargan de realizar estas mismas funcionalidades para mantener el móvil despierto a la escucha de posibles beacon para otros Sistemas Operativos, pero a diferencia de Apple esta funcionalidad no está integrada directamente en el Sistema Operativo por lo que no presenta el mismo nivel de control.


Asimismo, podemos afirmar que prácticamente la mayoría de las universidades cuentan con una disposición amplia en lo que se refiere a servicios y despliegue de medios. Como ejemplo, podemos coger la Universidad de la Laguna, la cual cuenta con una red WiFi con un rango de cobertura casi completo de sus instalaciones y una amplia carta de servicios disponibles para sus alumnos. Además cuenta con una serie de Beacons, que podrían ser instalados fácilmente en lugares estratégicos. 

Partiendo de esta base, procederemos a explorar posibles casos de uso para los beacons tomando el contexto universitario y del alumnado como referente:

\subsection{Guía a través del Campus de la Universidad}

Este caso de uso cubre la funcionalidad destacada de un beacon, el posicionamiento y guía tanto en exteriores como en interiores. 

Como interesados podríamos destacar: 

\begin{itemize}
\item Personal invitado a jornadas o eventos en instalaciones de la Universidad.
\item Alumnos de intercambio en programas internacionales.
\item Alumnos de nuevo acceso.
\item Personas con discapacidad.
\end{itemize}

\begin{figure}[h]
	\centering
	\includegraphics[width=\columnwidth]{locationMobileBeacon}
	\caption{Servicios de localización a través de la Universidad}
	\label{fig:beaconLocation}
\end{figure}

El funcionamiento sería el siguiente: 


El usuario transita por las inmediaciones del campus universitario. El usuario accede al sistema de navegación dentro de la aplicación. La aplicación le muestra entonces el camino mostrándole en todo momento su ubicación como un punto de color sobre el mapa del campus. Este mapa tiene marcados puntos de interés que contienen información de diferente tipo dependiendo del punto marcado: nombre, historia, página web, teléfono de contacto, trámites asociados... son algunos de los datos que podría mostrar. El mapa se va actualizando dependiendo de la posición del usuario permitiendo volver a la vista más alejada en cualquier momento para una visualización más general.


\subsection{Descarga automática de material}

Este caso de uso resultaría muy útil para personal lectivo y para estudiantes, los cuales accederían de manera más sencilla al material dado. También sería aplicable para ponentes de charlas los cuales no tendrían que colgar sus apuntes en alguna plataforma externa o llevarlos consigo en  un almacenamiento externo para compartirlo al finalizar.

El funcionamiento sería el siguiente: 


El profesor o ponente lleva consigo un beacon y sus estudiantes u oyentes tienen instalados en sus dispositivos la aplicación. El profesor es capaz de introducir en su aplicación con el perfil de profesor (el poniente con su correspondiente perfil), indicaciones del material a utilizar en el evento. El profesor carga consigo el pequeño dispositivo e indica a sus alumnos que conecten el bluetooth y abran la aplicación, al entrar en el rango, la aplicación pedirá permiso al alumno para descargarse el contenido indicado por el profesor. Si el alumno acepta, la aplicación pasaría a abrir  el contenido indicado por la página correspondiente.


\subsection{Acceso al parking y conteo de número de vehículos estacionados}


Este caso de uso proporcionaría información muy útil a los usuarios del parking de la universidad, informando del número coches estacionados en el parking y de las plazas restantes a ocupar en tiempo real.

El funcionamiento sería el siguiente: 


El personal de la Ull tendría la aplicación en su móvil, al acercarse a la barrera del parking, el usuario activaría el bluetooth de su móvil. El beacon por su parte registraría un nuevo punto entrando en el rango de acción del parking. La aplicación comprobaría que el usuario está autorizado a entrar en el parking y procedería a abrir la puerta del parking dejando entrar al vehículo. Cuando el vehículo saliese del rango del beacon por el rango interior, la aplicación registraría entonces un nuevo acceso al parking y contabilizaría otro vehículo dentro de parking. Al salir del parking el proceso sería el mismo, por lo tanto la aplicación sería capaz de informar al usuario de las plazas ocupadas en tiempo real.


\subsection{Gestión de eventos e información, entrada automática}

El funcionamiento sería el siguiente: 


Los alumnos transitan los interiores de la Universidad de camino a sus clases. Los beacons están desplegados en las inmediaciones de lugares de interés, tipo aularium, paraninfo, clases que se utilicen a modo de salas de reuniones o seminarios. Al pasar por las inmediaciones de estos lugares de interés, la aplicación sería capaz de proporcionar al usuario información de diversa índole: ponientes, tema de la charla, acceso, teléfono de contacto u otra información similar.  Al mismo tiempo, la aplicación también cuenta con un tablón donde se muestran posibles eventos futuros. Estos eventos pueden ser muy variados y corresponder a diferentes tipos de actividades. Al mismo tiempo se podría confirmar la entrada al evento en el caso de haberla, mediante un código de acceso identificativo generado al realizar el pago del evento.

\begin{figure}[H]
	\centering
	\includegraphics[width=\columnwidth]{BeaconEvent}
	\label{fig:eventBeacon}
\end{figure}

\subsection{Despacho del profesorado e información}

El funcionamiento podría abarcarse de dos maneras, por un lado, podría utilizarse para proveer a los alumnos de información acerca del grupo de despachos, aclarando que profesores tienen el despacho en la zona, horario de tutorías, correo electrónico de contacto, horario de corrección de exámenes, etc. De esta manera el alumno al acercarse a la zona sería capaz de saber información de todos los profesores, o si buscase a alguno en particular, la aplicación le daría la opción de elegir su nombre de una lista y simplemente comprobar si tiene su despacho en esa zona. 

Por otro lado, este caso de uso podría abarcarse para proporcionar una información adicional, comprobando si el profesor está en la zona en ese momento y se encuentra disponible. El profesor tendrá un perfil de la aplicación con un código identificativo que le distingue de los demás profesores. Estos datos se guardarían en un servicio externo, y el beacon sería el encargado de registrar las entradas y salidas de los profesores. En cuanto al estado de disponibilidad, sería un dato que actualizaría el profesor desde su perfil de profesor en la aplicación. Los alumnos recibirían estos estados desde su lado de la aplicación y serían capaces de saber cuando el profesor se encuentra disponible mediante la aplicación. 

\subsection{Información y descuentos para usuarios de la APP}

Este caso de uso no solo dependería de la universidad, sino de establecimientos comerciales interesados. La idea sería la siguiente: 

La universidad en colaboración con un establecimiento comercial le entrega un beacon. La aplicación contaría con un perfil para el dueño del establecimiento, donde sería capaz de introducir información que quiere que se muestre al usuario al pasar cerca de su establecimiento, mensajes de información, descuentos u ofertas especiales por ejemplo. El usuario al pasar por las inmediaciones del establecimeinto recibe en su aplicación una notificación del establecimiento con la información introducida por el dueño anteriormente. Al aceptar la notificación, el usuario podría ser redirigido a la página web del establecimiento para ver las ofertas. En cualquier caso el establecimiento ha conseguido captar la atención de un posible cliente, y el usuario se benefiaría de ofertas y descuentos. 

\subsection{Control de asistencia}

Este caso de uso podría ir ligado al de Descarga automática de material, el funcionamiento sería el siguiente: 

El alumno conectaría el bluetooth de su móvil al iniciar la clase, en este momento la aplicación detectaría los dispositivos con los identificadores de alu de los alumnos y los dejaría registrados a la clase en el horario establecido. Los profesores serían capaces en todo momento desde su perfil de profesor de consultar asistencia. Si lo unimos con la descarga automática de material que ya habíamos mencionado en apartados anteriores, proporcionaría comodidad tanto a alumnos como a profesores. Sin embargo, un impedimento podría ser el rango del beacon o la necesidad de activar el bluetooth ya que, si el alumno no tiene batería en el móvil, tendría que haber un método secundario. 

\subsection{Control de acceso a instalaciones}

En control de acceso a las aulas y edificios puede ser un tema abordable mediante el uso de estos dispositivos, los lectores de tarjetas pasarían a ser algo innecesario. El alumno simplemente tendría que activar el bluetooth cerca del punto de acceso, se comprobaría su identificador y procedería a darle a acceso o a informarle de su falta de permiso para acceder. El acceso a estos puntos podría quedar guardado en algún tipo de plataforma donde se monitorizen los accesos dependiendo de la seguridad de acceso al aula.

\subsection{Biblioteca Informativa}

Otro posible uso que se le podría dar a esta tecnología tiene que ver con las bibliotecas o lugares de almacenamiento de material. El alumno se acercaría a la biblioteca buscando un libro específico, en la aplicación estaría registrado la localización de los libros disponibles en los estantes, lo que le indicaría al alumno la posición del libro que busca. Para lugares amplios donde hay gran cantidad de material, incluso podría guiar al usuario como un punto por las instalaciones hasta llegar a su objetivo, informarle de si quedan ejemplares disponibles o de fecha prevista de entrada de algún material, de esta manera el alumno agilizaría su búsqueda en gran medida.

\begin{figure}[H]
	\centering
	\includegraphics[width=\columnwidth]{BibliotecaSalamanca}
	\caption{La biblioteca de la Universidad de Salamanca (USAL) contiene más de 1.000.000 de ejemplares lo que puede hacer dificil la localización de algunos títulos.}
	\label{fig:bibliotecaUSAL}
\end{figure}

\subsection{Actividades interactivas por el Campus, jornadas de acogida u otros eventos}

En un aspecto más recreativo, se podría tener en cuenta el uso de los beacons para organizar juegos o actividades divertidas para los alumnos. Estos eventos dependerían de los organizadores, pero podrían consistir en alguna actividad que implicase movimiento y colaboración. Los alumnos tendrían que registrarse con su identificador, una ruta a través del campus con adivinanzas o puzzles que tengan que ver con diferentes temáticas por ejemplo fomentaría a los alumnos a trabajar en equipo y utilizar su ingenio.Al mismo tiempo se podría aplicar algún tipo de recompensa para los ganadores, descuentos o bonos tramitados por medio de la aplicación, lo que fomentaría la participación estudiantil.

\subsection{Localización de transporte público, horarios e información de la parada}

Teniendo en cuenta los medios de transporte que utilizan los estudiantes, uno de los principales es el autobus. Este medio de transporte puede llegar a ser el día a día de muchos de los estudiantes que no cuentan con vehículo propio o que simplemente prefieren utilizar el autobus. 

Este caso de uso contempla lo siguiente: 


El estudiante llega a una parada de autobus y utilizando su dispositivo móvil, consulta los autobuses que van a pasar por la parada; la aplicación le permite obtener información de los diferentes autobuses, el itinerario y los minutos que faltan para que llege a la parada, asimismo también enlazaría con la web de la compañía de transporte para más información sobre las líneas y los itinerarios en caso de necesitar más información que la proporcionada por la aplicación.
% ---------------------------------------------------
%
% Trabajo de Fin de Grado. 
% Author: Laura Padrón Jorge. 
% Capítulo: La aplicacion BulletPoint. 
% Fichero: Cap4_TheApplication.tex
%
% ----------------------------------------------------
%

\chapter{La aplicación BulletPoint} \label{chap:laaplicacion} 

En este capítulo trataremos diversos temas relacionados con la aplicación, comenzaremos por definir posibles casos de uso en el ámbito universitario, tocaremos diversos temas relacionados incluyendo dificultades durante el desarrollo y acabaremos discutiendo posibles líneas futuras de desarrollo.

 
\section{Aplicaciones móviles en entornos universitarios}


\section {Posibles casos de uso de la tecnología beacon en entornos universitarios}

Actualmente uno de las posibilidades que se presentan para explotar esta tecnología se encuentra en las instituciones de enseñanza, las cuales podría utilizar los beacons para facilitar a sus alumnos, profesores y demás personal involucrado con sus actividades una serie de servicios de gran utilidad.

Sin embargo, para utilizar esta tecnología es necesario cumplir una serie de condiciones:

\begin{itemize}
\item Tener instalada la aplicación en su dispositivo móvil.
\item Tener activado el bluetooth.
\item La aplicación ha de estar despierta.
\item Las beacons han de estar desplegadas y configuradas correctamente en lugares clave donde el rango sea óptimo.
\end{itemize}

%leyenda, en dispositivos Iphone no es necesario tener activado el bluetooth ya que el SO se encarga de captar las señales BLE, aparte tampoco es necesario que la app este despierta que nuevamente el SO se encarga de despertar a la aplicación involucrada. Sin embargo Apple no ha desarrollado una Ibeacon física aún, aunque en un futuro, se espera que sus dispositivos móviles puedan funcionar como una beacon bidirecional.


Asimismo podemos afirmar que prácticamente hoy en día la mayoría de las universidades cuentan con una disposición amplia en los que se refiere a servicios y despliegue de medios. Como ejemplo podemos coger la Universidad de la Laguna, la cual cuenta con una red WiFi con un rango de cobertura casi completo de sus instalaciones y una amplia carta de servicios disponibles a sus alumnos por una serie de medios. Además cuenta con una serie de dispositivos beacons, que podrían ser instalados adecuadamente en lugares estratégicos. 

Partiendo de esta base, procederemos a explorar posibles casos de uso para los beacons en entornos universitarios tomando la ULL como referente:

\subsection{Guía a través del Campus de la ULL}

Este caso de uso cubre la funcionalidad destacada de un beacon, el posicionamiento y guia tanto en exteriores como en interiores. 

Como interesados podríamos destacar: 

\begin{itemize}
\item Personal invitado a jornadas o eventos en instalaciones de la ULL.
\item Alumnos de intercambio en programas internacionales.
\item Alumnos de nuevo acceso.
\item Personas con discapacidad.
\end{itemize}

\subsection{Descarga automática de material}

\subsection{}

\subsection{}

\subsection{}

\subsection{}

\subsection{}

\subsection{}

\section{4 Casos de uso elegidos}
%Arreglar una vez elija el caso de uso 


\section{Despliegue}



% ---------------------------------------------------
%
% Trabajo de Fin de Grado. 
% Author: Laura Padrón Jorge. 
% Capítulo: Conclusiones y lineas de trabajo futuras. 
% Fichero: Cap5_ConclusionsAndFutureLinesOfWork.tex
%
% ----------------------------------------------------
%

\chapter{Conclusiones y lineas de trabajo futuras} \label{chap:conclusiones} 

En este capítulo se presentaran las conclusiones a las que se ha llegado tras realizar este proyecto y discutiremos posibles lineas de trabajo futuras.

\subsection{Conclusiones}


\subsection{Lineas de trabajo futuros}



% ---------------------------------------------------
%
% Trabajo de Fin de Grado. 
% Author: Laura Padrón Jorge. 
% Capítulo: Presupuesto. 
% Fichero: Cap6_BudgetEstimations.tex
%
% ----------------------------------------------------
%

\chapter{Presupuesto y puesta en marcha} \label{chap:presupuesto} 

En este capítulo se expondrá de manera estimada las estimaciones de recursos necesarias para poner en práctica este despliegue, teniendo en cuenta el estado actual del proyecto. 

Actualmente el despliegue de la aplicación se podría separar en dos partes, por un lado tendríamos la compra de los distintos dispositivos y por otro lado hablaríamos del desarrollo y mantenimiento de la aplicación, por lo que se podría contabilizar de la siguiente manera: 

\begin{itemize}
\item Para el caso de uso de los autobuses sería necesario un beacon por parada, si solo se cubriesen las paradas dentro de la zona universitaria, como mucho podríamos estar hablando de unos 20 beacons, distribuidos entre anchieta y guajara.
\item Siguiendo con el caso de uso de los eventos, se podrían calcular unas 10 zonas de eventos principales, nuevamente a distribuir entre guajara y anchieta. Zonas como el paraninfo, el aularium, aulas magnas, bibliotecas o lugares destinados específicamente para la realización de eventos.
\item Los casos de uso que dependen de la localización dependerían de las zonas que habría que controlar, por lo que habría que hacer un análisis de cada recinto en  el caso del parking, aulas para el módulo de asistencia y edificios o exteriores para el módulo de guía.
\end{itemize}


Teniendo en cuenta que cada dispositivo de Aruba cuesta entre 20 a 30 euros.

 
\begin{itemize}
\item Para cubrir el primer caso de uso sería necesaria una inversión de 400 a 600 euros con lo que se cubrirían 20 paradas de autobus.
\item Para las 10 zonas de eventos estaríamos hablando de unos 200 a 300 euros, sería posible añadir más beacons si fuera necesario.
\item Los casos de uso que dependen de la localización son muy relativos, los parkings son cerca de unos 8, teniendo en cuenta que mínimo necesitaríamos unos 3 beacons por recinto, tendríamos un total de 480 a 720 euros, en el caso de la asistencia y guía podríamos calcular para todos los edificios de la ULL unos 100 beacons, con lo que tendríamos cubiertas las facultades en su mayoría con un importe entre 2000 a 3000 eruos.
\end{itemize}


Juntando todo esto nos quedaría un presupuesto de unos 3850 euros de media, y quedarían cubiertas los edificios, parkings y paradas de autobus cercanas al campus. Aparte de los dispositivos habría que añadir el mantenimiento de la aplicación para añadir los datos de los diferentes beacons, incluyendo un servidor para almacenar los datos y realizar las consultas. 



%\newpage{\pagestyle{empty}}
%\thispagestyle{empty}

%\chapter{Presupuesto}
%\label{chapter:Presupuesto}

%\input{cap7.tex}

%%%%%%%%%%%%%%%%%%%%%%%%%%%%%%%%%%%%%%%%%%%%%%%%%%%%%%%%%%%%%%%%%%%%%%%%%%%%%%%

%%%%%%%%%%%%%%%%%%%%%%%%%%%%%%%%%%%%%%%%%%%%%%%%%%%%%%%%%%%%%%%%%%%%%%%%%%%%%%%
%\newpage{\pagestyle{empty}}
%\thispagestyle{empty}
%\begin{appendix}
%
%\chapter{Título del Apéndice 1}
%\label{appendix:1}
%\input{apendice1.tex}
%
%\chapter{Título del Apéndice 2}
%\label{appendix:2}
%\input{apendice2.tex}
%
%\end{appendix}
%%%%%%%%%%%%%%%%%%%%%%%%%%%%%%%%%%%%%%%%%%%%%%%%%%%%%%%%%%%%%%%%%%%%%%%%%%%%%%%
\addcontentsline{toc}{chapter}{Bibliografía}
\bibliographystyle{plain}
\renewcommand{\bibname}{Bibliografía}   %  Para que no aparezca Índice de figuras
\bibliography{bibliografia}

%%%%%%%%%%%%%%%%%%%%%%%%%%%%%%%%%%%%%%%%%%%%%%%%%%%%%%%%%%%%%%%%%%%%%%%%%%%%%%%

\end{document}
