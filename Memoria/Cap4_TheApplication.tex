% ---------------------------------------------------
%
% Trabajo de Fin de Grado. 
% Author: Laura Padrón Jorge. 
% Capítulo: La aplicacion BulletPoint. 
% Fichero: Cap4_TheApplication.tex
%
% ----------------------------------------------------
%

\chapter{La aplicación BulletPoint} \label{chap:laaplicacion} 

En este capítulo trataremos diversos temas relacionados con la aplicación, comenzaremos por definir posibles casos de uso en el ámbito universitario, tocaremos diversos temas relacionados incluyendo dificultades durante el desarrollo y acabaremos discutiendo posibles líneas futuras de desarrollo.

 
\section{Aplicaciones móviles en entornos universitarios}


\section {Posibles casos de uso de la tecnología beacon en entornos universitarios}

Actualmente uno de las posibilidades que se presentan para explotar esta tecnología se encuentra en las instituciones de enseñanza, las cuales podría utilizar los beacons para facilitar a sus alumnos, profesores y demás personal involucrado con sus actividades una serie de servicios de gran utilidad.

Sin embargo, para utilizar esta tecnología es necesario cumplir una serie de condiciones:

\begin{itemize}
\item Tener instalada la aplicación en su dispositivo móvil.
\item Tener activado el bluetooth.
\item La aplicación ha de estar despierta.
\item Las beacons han de estar desplegadas y configuradas correctamente en lugares clave donde el rango sea óptimo.
\end{itemize}

%leyenda, en dispositivos Iphone no es necesario tener activado el bluetooth ya que el SO se encarga de captar las señales BLE, aparte tampoco es necesario que la app este despierta que nuevamente el SO se encarga de despertar a la aplicación involucrada. Sin embargo Apple no ha desarrollado una Ibeacon física aún, aunque en un futuro, se espera que sus dispositivos móviles puedan funcionar como una beacon bidirecional.


Asimismo podemos afirmar que prácticamente hoy en día la mayoría de las universidades cuentan con una disposición amplia en los que se refiere a servicios y despliegue de medios. Como ejemplo podemos coger la Universidad de la Laguna, la cual cuenta con una red WiFi con un rango de cobertura casi completo de sus instalaciones y una amplia carta de servicios disponibles a sus alumnos por una serie de medios. Además cuenta con una serie de dispositivos beacons, que podrían ser instalados adecuadamente en lugares estratégicos. 

Partiendo de esta base, procederemos a explorar posibles casos de uso para los beacons en entornos universitarios tomando la ULL como referente:

\subsection{Guía a través del Campus de la ULL}

Este caso de uso cubre la funcionalidad destacada de un beacon, el posicionamiento y guia tanto en exteriores como en interiores. 

Como interesados podríamos destacar: 

\begin{itemize}
\item Personal invitado a jornadas o eventos en instalaciones de la ULL.
\item Alumnos de intercambio en programas internacionales.
\item Alumnos de nuevo acceso.
\item Personas con discapacidad.
\end{itemize}

\subsection{Descarga automática de material}

\subsection{}

\subsection{}

\subsection{}

\subsection{}

\subsection{}

\subsection{}

\section{4 Casos de uso elegidos}
%Arreglar una vez elija el caso de uso 


\section{Despliegue}


