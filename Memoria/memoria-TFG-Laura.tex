%
% ---------------------------------------------------
%
% Trabajo Fin de Grado:
% Author: Laura Padrón Jorge <gonzalezsuarezivan@gmail.com>
% Author: F. de Sande fsande@ull.es
% Fichero: main.tex
%
% ----------------------------------------------------
%
\documentclass[spanish,a4paper,14pt,oneside]{extreport}
%\documentclass[a4paper, twoside, 12pt]{book}
\usepackage[a4paper]{geometry}
\usepackage[spanish]{babel}
\usepackage[utf8]{inputenc}
%\usepackage{lscape}
\usepackage{pdflscape}
%%%%%%%%%%%%%%%%%%%%%%%%%%%%%%%%%%%%%%%%%%%%%%%%%%%%%%%%%%%%%%%%%%%%%%%%%%%%%%%%%%%%%%%%%%%%
% Next 3+3 lines select PDF or PS output (comment as apropriate)
% To switch from PDF and PS comment/uncomment here and change Makefile
\usepackage[pdftex]{color}
\usepackage[pdftex]{graphicx}
\graphicspath{{images/}}
%\usepackage[dvips]{color}
%\usepackage[dvips]{graphicx}
\usepackage{epsfig}
%\graphicspath{{images/eps/}}
%Añadidos BulletPoint
\usepackage{floatrow}
%%%%%%%%%%%%%%%%%%%%%%%%%%%%%%%%%%%%%%%%%%%%%%%%%%%%%%%%%%%%%%%%%%%%%%%%%%%%%%%%%%%%%%%%%%%%
\usepackage{algorithmic}
\usepackage[pdftex=true,colorlinks=false,urlcolor=blue,plainpages=false,pagebackref=true,citecolor=red]{hyperref} %hiperenlaces y backcites 
%%%%%%%%%%%%%%%%%%%%%%%%%%%%%%%%%%%%%%%%%%%%%%%%%%%%%%%%%%%%%%%%%%%%%%%%%%%%%%%%%%%%%%%%%%%%
% Comandos para escribir "siempre igual"
\newcommand{\BulletPoint}{\texttt{BulletPoint:{ Innovando con Beacons}}}

%%% Traducimos el pseudocodigo
\renewcommand{\algorithmicwhile}{\textbf{mientras}}
\renewcommand{\algorithmicend}{\textbf{fin}}
\renewcommand{\algorithmicdo}{\textbf{hacer}}
\renewcommand{\algorithmicif}{\textbf{si}}
\renewcommand{\algorithmicthen}{\textbf{entonces}}
\renewcommand{\algorithmicrepeat}{\textbf{repetir}}
\renewcommand{\algorithmicuntil}{\textbf{hasta que}}
\renewcommand{\algorithmicelse}{\textbf{en otro caso}}
\renewcommand{\algorithmicfor}{\textbf{para}}

%%%%%%%%%%%%%%%%% Creamos un entorno para listar código fuente %%%%%%%%%%%%%%%
\newenvironment{sourcecode}
{\begin{list}{}{\setlength{\leftmargin}{1em}}\item\scriptsize\bfseries}
{\end{list}}

\newenvironment{littlesourcecode}
{\begin{list}{}{\setlength{\leftmargin}{1em}}\item\tiny\bfseries}
{\end{list}}

\newenvironment{summary}
{\par\noindent\begin{center}\textbf{Abstract}\end{center}\begin{itshape}\par\noindent}
{\end{itshape}}

\newenvironment{keywords}
{\begin{list}{}{\setlength{\leftmargin}{1em}}\item[\hskip\labelsep \bfseries Keywords:]}
{\end{list}}

\newenvironment{palabrasClave}
{\begin{list}{}{\setlength{\leftmargin}{1em}}\item[\hskip\labelsep \bfseries Palabras clave:]}
{\end{list}}


%%%%%%%%%%%%%%%%%%%%%%%%%%%%%%%%%%%%%%%%%%%%%%%%%%%%%%%%%%%%%%%%%%%%%%%%%%%%%%%
\definecolor{marron}       {rgb}{0.496, 0.203, 0.152}
\definecolor{verde-claro}  {rgb}{0.625, 0.734, 0.199}
\definecolor{oscuro}       {rgb}{0.187, 0.141, 0.285}
\definecolor{gris}     	   {rgb}{0.500, 0.500, 0.500}
\definecolor{bgd-listings} {rgb}{0.999, 0.999, 0.900}
\definecolor{gray97}{gray}{.97}
\definecolor{gray75}{gray}{.75}
\definecolor{gray45}{gray}{.45}
\definecolor{gray}{gray}{.45}
%%%%%%%%%%%%%%%%%%%%%%%%%%%%%%%%%%%%%%%%%%%%%%%%%%%%%%%%%%%%%%%%%%%%%%%%%%%%%%%%%%%%%%%%%%%%
%%% Code Listings
%\usepackage{listings} 
%\lstloadlanguages{python,C}
\definecolor{Brown}{cmyk}{0,0.81,1,0.60}
\definecolor{OliveGreen}{cmyk}{0.64,0,0.95,0.40}
\definecolor{CadetBlue}{cmyk}{0.62,0.57,0.23,0}
\definecolor{lightlightgray}{gray}{0.9}
%%%%%%%%%%%%%%%%%%%%%%%%%%%%%%%%%%%%%%%%%%%%%%%%%%%%%%%%%%%%%%%%%%%%%%%%%%%%%%%%%%%%%%%%%%%
%Evitar partir palabras al final de la línea
%\hyphenpenalty=10000
%\tolerance=1000
%%%%%%%%%%%%%%%%%%%%%%%%%%%%%%%%%%%%%%%%%%%%%%%%%%%%%%%%%%%%%%%%%%%%%%%%%%%%%%%%%%%%%%%%%%%%
% Para listados de código
\usepackage{listings}
\lstloadlanguages{C}

% Definiendo colores para los listados de código fuente - Univ. Deusto
\definecolor{violet}{rgb}{0.5,0,0.5}
\definecolor{navy}{rgb}{0,0,0.5}
\definecolor{hellgelb}{rgb}{1,1,0.8}
\definecolor{colKeys}{rgb}{0,0,1}
\definecolor{colIdentifier}{rgb}{0,0,0}
\definecolor{colComments}{rgb}{1,0,0}
\definecolor{colString}{rgb}{0,0.5,0}

%\lstset{morekeywords={pragma copy\_in copy\_out copy omp parallel private reduction shared hicuda loop\_partition over\_tblock over\_thread}}
\lstset{
        float=tbhp,
		    language = Java,
				morekeywords={llc,reduction_type,nc_result,
				              hicuda,global,alloc,shape,kernel,thread,loop_partition,tblock,over_tblock,over_thread,kernel_end,copyout,free,
											data,region,
											task,input,inout,output,
				              pragma,omp,parallel,reduction,private,shared,target,device,copy_in,copy_out,
				              acc,kernels,loop,copyin,copy,pcopy,pcopyin,collapse,gang,worker,independent},
				%\emph      ={omp,parallel,reduction,private,shared},
				emphstyle=\textbf,
        %basicstyle=\ttfamily\tiny,
        basicstyle=\ttfamily\scriptsize,
        identifierstyle=\color{colIdentifier},
        keywordstyle=\color{colKeys},
        stringstyle=\color{colString},
        commentstyle=\color[rgb]{0.133,0.545,0.133},
        columns=flexible,
        tabsize=4,
        frame=single,
        extendedchars=true,
        showspaces=false,
        showstringspaces=false,
        numbers=left,
        numberstyle=\tiny,
        breaklines=true,
        backgroundcolor=\color{lightlightgray},
        breakautoindent=true,
        captionpos=b
}

%\renewcommand{\lstlistingname}{Listing} % Los títulos de los códigos insertados se denotan con Ejemplo...   

% Otro formato más bonito para código fuente
\newcommand{\codigofuente}[3]{%
  \lstlisting[language=#1,caption={#2}]{#3}%
}
%%%%%%%%%%%%%%%%%%%%%%%%%%%%%%%%%%%%%%%%%%%%%%%%%%%%%%%%%%%%%%%%%%%%%%%%%%%%%%%
\begin{document}
\renewcommand{\lstlistingname}{Listado}% Listing -> Listado de código
%%%%%%%%%%%%%%%%%%%%%%%%%%%%%%%%%%%%%%%%%%%%%%%%%%%%%%%%%%%%%%%%%%%%%%%%%%%%%%%
% First Page
%%%%%%%%%%%%%%%%%%%%%%%%%%%%%%%%%%%%%%%%%%%%%%%%%%%%%%%%%%%%%%%%%%%%%%%%%%%%%%%

\pagestyle{empty}
\thispagestyle{empty}


\newcommand{\HRule}{\rule{\linewidth}{1mm}}
\setlength{\parindent}{0mm}
\setlength{\parskip}{0mm}

\vspace*{\stretch{0.5}}

\begin{center}
\includegraphics[scale=0.8]{images/logo_vertical}\\[10mm]
{\Huge Trabajo de Fin de Grado}
\end{center}

\HRule
\begin{flushright}
        {\Huge \BulletPoint{}} \\[2.5mm]
        {\Large Laura Padrón Jorge} \\[5mm]


\end{flushright}
\HRule
\vspace*{\stretch{2}}
\begin{center}
  \Large La Laguna, \today
\end{center}

\setlength{\parindent}{5mm}

%%%%%%%%%%%%%%%%%%%%%%%%%%%%%%%%%%%%%%%%%%%%%%%%%%%%%%%%%%%%%%%%%%%%%%%%%%%%%%%
% Signature page (add the official stamp)
%%%%%%%%%%%%%%%%%%%%%%%%%%%%%%%%%%%%%%%%%%%%%%%%%%%%%%%%%%%%%%%%%%%%%%%%%%%%%%%
\newpage
%\cleardoublepage
\thispagestyle{empty}

D. {\bf Francisco de Sande González}, con N.I.F. 42.067.050-G
profesor
Titular de Universidad
adscrito al Departamento
de Ingeniería Informática y de Sistemas
de la Universidad de La Laguna, como tutor

\bigskip

\bigskip
\bigskip
{\bf C E R T I F I C A}

\bigskip
\bigskip
\bigskip
Que la presente memoria titulada:

\bigskip
``{\it \BulletPoint{}}''

\bigskip
\bigskip
\bigskip
%Cambiar
\noindent ha sido realizada bajo su dirección por Dñ. {\bf Laura Padron Jorge},
con N.I.F. 79.890.251-W

\bigskip
\bigskip

Y para que así conste, en cumplimiento de la legislación vigente y a los efectos
oportunos firman la presente en La Laguna a \today

%\cleardoublepage
\newpage
%%%%%%%%%%%%%%%%%%%%%%%%%%%%%%%%%%%%%%%%%%%%%%%%%%%%%%%%%%%%%%%%%%%%%%%%%%%%%%%
\thispagestyle{empty}

{ \flushright

\begin{LARGE}
Agradecimientos
\end{LARGE}

\hspace{3mm}

\begin{large}


\hspace{3mm}

Mis agradecimientos al profesor Francisco de Sande González por su gran labor como tutor de este proyecto, orientando este trabajo, compartiendo su conocimiento y exigiendo siempre lo mejor. 

Asimismo me gustaría agradecer a la Universidad de La Laguna, a los Servicios TIC y a Don Juan Carlos Hernández Perdomo su colaboración en este proyecto y  la puesta a disposición de las beacons, sin las cuales habría sido imposible el desarrollo de este proyecto. 

Gracias también a Don Alberto Morales por su tiempo y dedicación en la explicación de los requisitos técnicos necesarios para la implantación de los beacons en el sistema de parking de la Ull, objecto de estudio en el desarrollo de este trabajo.

\end{large}

}

%%%%%%%%%%%%%%%%%%%%%%%%%%%%%%%%%%%%%%%%%%%%%%%%%%%%%%%%%%%%%%%%%%%%%%%%%%%%%%%%%
\newpage

\begin{huge}
Licencia
\end{huge}

\bigskip
%* Si quiere permitir que se compartan las adaptaciones de tu obra mientras se comparta de la misma manera
%y NO quieres permitir usos comerciales de tu obra indica:

\begin{center}
\includegraphics[scale=1.5]{images/by-nc-sa_88x31}\\[10mm]
{\Large \copyright~Esta obra está bajo una licencia de Creative Commons Reconocimiento-NoComercial-CompartirIgual 4.0 Internacional.
}
\end{center}


%%%%%%%%%%%%%%%%%%%%%%%%%%%%%%%%%%%%%%%%%%%%%%%%%%%%%%%%%%%%%%%%%%%%%%%%%%%%%%%
\newpage  %\cleardoublepage
\begin{abstract}
{\em

En este documento se desarrolla el trabajo de investigación y desarrollo de la alumna durante el proceso de construcción de una aplicación para dispositivos móviles Android, utilizando una de las tecnologías más recientes en el mercado actual, los Beacons.

\bigskip
Partimos de los conocimientos en \textit{Java} obtenidos en la asignatura: \textit{'Diseño arquitectónico y patrones'} cursada en el 
itinerario de Ingeniería del Software, (impartida en el tercer curso del Grado en Ingeniería Informática de la Universidad de La Laguna), que han sentado una base previa de conocimientos a partir de los cuales la alumna ha desarrollado su aplicación.

\bigskip
Mediante el desarrollo de este trabajo, la alumna ha conseguido adquirir independencia en su trabajo, visión y planificación, desarrollando tareas de investigación, documentación, desarrollo y despliegue, que han dado como resultado la obtención de conocimientos durante el proceso de trabajo.

\bigskip 
También se ha investigado una tecnología reciente en el mercado "los Beacons", acercando a la alumna a esta nueva tecnología,  que si bien aún no tiene un impacto muy grande, en unos años se espera que se empiece a utilizar con naturalidad en distintos ámbitos: Turismo, Comercio, Enseñanza, etc.

}
\begin{palabrasClave}
Aplicaciones Android, Beacons.
\end{palabrasClave}

\end{abstract}
%%%%%%%%%%%%%%%%%%%%%%%%%%%%%%%%%%%%%%%%%%%%%%%%%%%%%%%%%%%%%%%%%%%%%%%%%%%%%%%

%%%%%%%%%%%%%%%%%%%%%%%%%%%%%%%%%%%%%%%%%%%%%%%%%%%%%%%%%%%%%%%%%%%%%%%%%%%%%%%
\newpage  %\cleardoublepage
\begin{summary}
{\em

The aim of the Project has been the development of an application for Android devices which will be using beacon technology for some of its main features.

\bigskip
Based on the knowledge of Java obtained in the subject: ''Architectural Design and Patterns'' taken in the
Software Engineering Branch (Given in third year of Computing Engineering Degree from ''La Universidad de La Laguna''). 
in this work we have acquired the basic knowledge needed to
develop Android applications introducing us in the development of applications related to Beacon Technology.

\bigskip
Moreover, the student has learned independency in her work and gained vision and planification aptitudes, developing differents labours of investigation, documentation, development and deployment, that have come to give her vast knowledge during the development of this project.

\bigskip
Apart from all this, it's of great value for me to get to know this new Beacon technolgy. I have investigated and learned from this new technology, which at the moment is not well knownl, in the close future it is expected to get more attention in differents sectors, such as Tourism, Trading or Learning.
}

\begin{keywords}
Application for Android, Beacons.
\end{keywords}

\end{summary}
%%%%%%%%%%%%%%%%%%%%%%%%%%%%%%%%%%%%%%%%%%%%%%%%%%%%%%%%%%%%%%%%%%%%%%%%%%%%%%%

%%%%%%%%%%%%%%%%%%%%%%%%%%%%%%%%%%%%%%%%%%%%%%%%%%%%%%%%%%%%%%%%%%%%%%%%%%%%%%%
\newpage{\pagestyle{empty}}
\thispagestyle{empty}

%%%%%%%%%%%%%%%%%%%%%%%%%%%%%%%%%%%%%%%%%%%%%%%%%%%%%%%%%%%%%%%%%%%%%%%%%%%%%%%


\pagestyle{myheadings} %my head defined by markboth or markright
% No funciona bien \markboth sin "twoside" en \documentclass, pero al
% ponerlo se dan un montón de errores de underfull \vbox, con lo que no se
% ha puesto.
\markboth{Laura Padrón Jorge}{BulletPoint}

%%%%%%%%%%%%%%%%%%%%%%%%%%%%%%%%%%%%%%%%%%%%%%%%%%%%%%%%%%%%%%%%%%%%%%%%%%%%%%%
%Numeracion en romanos
\renewcommand{\thepage}{\roman{page}}
\setcounter{page}{1}

%%%%%%%%%%%%%%%%%%%%%%%%%%%%%%%%%%%%%%%%%%%%%%%%%%%%%%%%%%%%%%%%%%%%%%%%%%%%%%%

\tableofcontents

%%%%%%%%%%%%%%%%%%%%%%%%%%%%%%%%%%%%%%%%%%%%%%%%%%%%%%%%%%%%%%%%%%%%%%%%%%%%%%%
\newpage{\pagestyle{empty}}

\listoffigures

%%%%%%%%%%%%%%%%%%%%%%%%%%%%%%%%%%%%%%%%%%%%%%%%%%%%%%%%%%%%%%%%%%%%%%%%%%%%%%%
\newpage{\pagestyle{empty}}

%\listoftables

%%%%%%%%%%%%%%%%%%%%%%%%%%%%%%%%%%%%%%%%%%%%%%%%%%%%%%%%%%%%%%%%%%%%%%%%%%%%%%%
\newpage{\pagestyle{empty}}

%%%%%%%%%%%%%%%%%%%%%%%%%%%%%%%%%%%%%%%%%%%%%%%%%%%%%%%%%%%%%%%%%%%%%%%%%%%%%%%
%Numeracion a partir del capitulo I
\renewcommand{\thepage}{\arabic{page}}
\setcounter{page}{1}


% ==========================================================
% --------               Capítulos                ----------
% --------    Estan en el directorio capitulos/   ----------
% ==========================================================
% ---------------------------------------------------
%
% Proyecto de Final de Carrera: 
% Author: Laura Padron Jorge <alu0100703511@ul.edu.es>
% Introducción
% Fichero: Prologo.tex
%
% ----------------------------------------------------

\chapter*{Introducción}
\addcontentsline{toc}{chapter}{Introducción} 

Este documento comprende el trabajo de investigación y desarrollo realizado por el autor en la consecución de su Trabajo de Fin de Grado (TFG), con el que pondrá fin a sus estudios del Grado en Ingeniería Informática cursados en la Escuela Técnica Superior de Ingeniería Informática (ETSII) de la Universidad de la Laguna (ULL).

%
% ---------------------------------------------------
%
% Proyecto de Final de Carrera:
% Author: Laura Padrón Jorge <alu0100703511@ull.edu.es>
% Capítulo: Objetivos 
% Fichero: Cap1_Goals.tex
%
% ----------------------------------------------------
%


\chapter{Objetivos} \label{chap:Objetivos}  

Este TFG tiene los siguientes objetivos principales:

	
\begin{itemize}
\item  	Por un lado se pretende ampliar los conocimientos en tecnologías móviles en el sistema operativo \textit{Android} \cite{URL::Android} y en el desarrollo de aplicaciones para este sistema operativo.
\item Por otro lado, también se pretende que la alumna se familiarice con el uso de herramientas de control de versiones utilizando GitHub \textit{Github} \cite{URL::Github} y de edición de textos técnicos utilizando \textit{LaTeX}  \cite{URL::LaTeX}.
\item Otro objetivo presente en este TFG es que la alumna investigue y profundice en una tecnología reciente en el mercado, los \textit{Beacons} \cite{URL::Beacon}.
\item  Por último, se espera que la estudiante obtenga independencia en su trabajo, visión y planificación poniendo en marcha tareas de investigación, documentación, desarrollo y despliegue utilizando tanto los conocimientos adquiridos durante la carrera, como aquellos que se irán aprendiendo durante el progreso de este trabajo.
\end{itemize}
% ---------------------------------------------------
%
% Proyecto de Final de Carrera:
% Author: Laura Padrón Jorge <alu0100703511@ull.edu.es>
% Capítulo: Herramientas de Desarrollo utilizadas 
% Fichero: Cap2_SoftwareTools.tex
%
% ----------------------------------------------------
%


\chapter{Herramientas de Desarrollo utilizadas} \label{chap:HerramientasSoftware}

Este capitulo tiene como objetivo presentar las distintas Herramientas Software empleadas por la alumna en el desarrollo de su TFG.

\section{Android Studio}


\section{Balsamic Mock Up}


\section{Eclipse + extensión ADT}

\section{LaTex}

\section{Github}

\section{Aruba Editor}


% ---------------------------------------------------
%
% Trabajo de Fin de Grado. 
% Author: Laura Padrón Jorge. 
% Capítulo: Tecnologías utilizadas en el Trabajo de Fin de Grado. 
% Fichero: Cap3_Technology.tex
%
% ----------------------------------------------------
%

\cleardoublepage
\chapter{Tecnologías} \label{chap:polytopes} %Cambiar por el label adecuado. 

En este capítulo se habla de las principales tecnologías que han sido utilizadas durante la elaboración de este TFG.

\section{Beacons}

\textit{''Beacons''} \cite{URL::Beacons} cuya traducción del inglés equivaldría a \textit{''balizas''} o \textit{''faros''}, es una tecnología emergente que desde algunos años se está intentando abrir paso en el mercado. Como su propio nombre indica, estos dispositivos intentan dar una mejor solución al posicionamiento en interiores, siendo un mecanismo de guía en lugares donde otras tecnologías, como el GPS o el Wifi dejan de funcionar o resultan imprecisas. 


Sin embargo, estos no son los únicos usos de los beacons, actualmente muchas empresas están ampliando sus usos a otros campos, y el diseño de estos beacons se está presentando en tamaños tan pequeños y con un tiempo de funcionamiento tan elevado, que se pueden desplegar prácticamente en cualquier lugar sin dificultades.

\begin{figure}[h]
	\left
	\includegraphics[width=\columnwidth]{estimoteBeacon.png}
	\caption{Uno de los beacons de la compañia Estimote}
	\label{fig:estimote_Beacon}
\end{figure}

\begin{figure}[h]
	\middle
	\includegraphics[width=\columnwidth]{estimoteBeaconSticker.png}
	\caption{Uno de los beacons de la compañia Estimote en formato Pegatina}
	\label{fig:estimote_Beacon_Sticker}
\end{figure}

\begin{figure}[h]
	\right
	\includegraphics[width=\columnwidth]{BluesenseBeacon.png}
	\caption{Uno de los beacons de la compañia BlueSense}
	\label{fig:bluesense_Beacon}
\end{figure}

A continuación se intentará responder a las preguntas más frecuentes que nos pueden surgir con respecto a esta tecnología:


\begin{itemize}
\item ¿Qué es un Beacon?
\item ¿Cómo funcionan estos dispositivos?
\item ¿Qué rango de alcance poseen?
\item ¿Con qué dispositivos móviles son compatibles? 
\item ¿Qué ventajas y desventajas tienen con respecto a otras tecnologías?
\item ¿Qué usos se le ha dado a esta tecnología hasta ahora?
\item ¿Qué empresas trabajan con esta tecnología?
\end{itemize}

\subsection{¿Qué es un Beacon?}

Para los que no hayan oido este término, en el marco en el que nos movemos, hace referencia a un pequeño dispositivo (sus tamaños varían de uno a otro, pero siempre de tamaño reducido) que emite señales de onda corta utilizando la tecnología Bluetooth \cite{URL::Bluetooth}. Estas señales contienen una pequeña cantidad de información y son recibidas por dispositivos móviles con tecnología Bluetooth dentro de un rango de cobertura variable dependiendo del propio dipositivo. Normalmente, la fuerza de esta señal y su frecuencia son configurables.

%Foto de algo que tenga que ver con el Bluetooth 

El funcionamiento de un beacon es sencillo: El beacon emite una señal ininterrumpida que es captada por los dispositivos móviles dentro de su radio de cobertura, esta señal contiene información capaz de definir una localización, detectar y localizar otros dispositivos. A continuación la señal es captada por una aplicación movil previamente instalada, que dependiendo de la señal recibida, puede lanzar una acción en dicho dispositivo.


Hay que tener en cuenta que esta señal es unilateral: los beacons son capaces de enviar señales pero no están preparados para recibirlas. También hay que tener en cuenta, que la mayoría de las beacon actuales en el mercado transmiten información preconfigurada, confiando en la aplicacion móvil para utilizar la información; sin embargo es muy posible que esto cambie en un futuro, ampliando las posibilidades de los beacons.

\subsection{¿Como funcionan estos dispositivos?}

Los beacons usan Bluetooth Low Energy (BLE) \cite{URL::BluetoothLowEnergy}, una version del protocolo Bluetooth diseñada para usar mucha menos energía y enviar menos información. Los beacons funcionan con baterías cuyo tiempo de vida depende de la configuración establecida, teniendo en cuenta la emisión de la señal (fuerza y frencuencia) y tiempo de hibernación. Sus tiempos de vida son variables, pudiendo durar desde un mes hasta varios años. 



%Foto de una beacon de aruba por dentro con las baterías a la vista.

Independientemente de lo que se pueda pensar, los beacons en si mismas no transmiten información significativa, transmiten identificadores cortos junto con información customizable breve, que son interpretadas por una aplicación que sabe lo que tiene que hacer con esa información y que es la que se encarga de procesar la información y realizar una acción pertinente.

Este identificador se divide en tres partes: 

\begin{itemize}
\item \textit{''UUID''} \cite{URL::UUID} : corresponde con una ID dada por el vendedor e identifica el beacon en cuestión.
\item ID Superior : customizables y utilizadas con un significado específico que puede identificar una acción o parámetro. 
\item ID Inferior: customizables y utilizadas al igual que la superior con un significado específico que se puede usar para identificar una acción o parámetro.
\end{itemize}

%Imagen del numero identificativo de la Beacon.

\subsection{¿Qué rango de alcance poseen?}

Actualmente los beacons presentan un rango de aproximadamente 70 metros sin obstáculos, esta demostrado que este rango disminuye significativamente al atravesar paredes de metal o ladrillo, otros materiales disminuyen en menor medida el rango. 

Los beacons además trabajan con tres rangos de distancia principalmente: 

\begin{itemize}
\item Lejos: diseñado para que el dispositivo móvil pueda lanzar una acción cuando estás en el rango exterior de un beacon, acabas de entrar en el rango del beacon.
\item Cerca: diseñado para que el dispositivo móvil pueda lanzar una acción cuando estás en el rango interior del beacon. 
\item Inmediato: Diseñado para que el dispositivo móvil pueda lanzar una acción cuando te encuentres manejando el beacon, la posición del beacon cambia.
\end{itemize}

\subsection{¿Con qué dispositivos funcionan?}

Las beacons son compatibles con todos los dispositivos que soporten Bluetooth Low Energy, pero para que las señales de los beacons sean detectadas por tu dispositivo, se ha de tener activado el Bluetooth. 


En dispositivos con IOS7 \cite{URL::IOS7} o superior, el dispositivo puede estar constantemente buscando dispositivos BLE y despertar a las aplicaciones implicadas cuando entran en el rango de los beacons, incluso estando cerradas las aplicaciones.


En dispositivos Android \cite{URL::Android} el sistema operativo no está preparado para escanear dispositivos BLE, por lo que son las aplicaciones las que se tienen que encargar de escanear las proximidades buscando beacons, esto supone que las aplicaciones tienen que estar funcionando, despiertas (aunque sea en segundo plano).

Por último en dispositivos Windows Phone \cite{URL:WindowsPhone} o Blackberry \cite{URL:Blackberry} existen diferentes niveles de compatibilidad pero en los que soportan BLE, en funcionamiento es similar al de los dispositivos Android. 

\subsection{¿Qué ventajas y desventajas tienen con respecto a otras tecnologías?}

A la hora de hablar de los beacons existen una serie de ventajas pero también podemos encontrar algunas desventajas que iremos detallando a continuación. 


Las principales ventajas que se distinguen a la hora de hablar de las beacons son las siguientes: 

\begin{itemize}
\item A diferencia de la tecnología GPS, la activación del bluetooth consume mucho menos batería. 
\item Es una tecnología que puede ser dependiente de la red de datos. 
\item A diferencia de la tecnología GPS, sigue funcionando con gran precisión en el interior de los edificios.
\end{itemize}

En cuanto a las desventajes que nos podemos encontrar destacamos:

\begin{itemize}
\item Dependen de aplicaciones instaladas en el dispositivo móvil para funcionar. 
\item Es necesario tener el bluetooth activado, lo que consume batería en el tiempo. 
\item Su utilidad depdende de la voluntad de terceros de utilizar estos dispositivos, configurarlos y distribuir las aplicaciones.
\end{itemize}

\subsection{¿Qué usos se le ha dado a esta tecnología hasta ahora?}



\subsection{¿Qué empresas trabajan con esta tecnología?}



%%%%%%%%%%%%%%%%%%%%%%%%%%%%%%%%%%%%% Figure %%%%%%%%%%%%%%%%%%%%%%%%%%%%%%%%%%%%%%%%%%%%%
%\begin{figure}[h]
%	\centering
%	\includegraphics[width=\columnwidth]{poly.png}
%	\caption{Representación poliédrica de dos bucles anidados}
%	\label{fig:polytope_1}
%\end{figure}
%%%%%%%%%%%%%%%%%%%%%%%%%%%%%%%%%%%%%%%%%%%%%%%%%%%%%%%%%%%%%%%%%%%%%%%%%%%%%%%%%%%%%%%%%%

%La Matriz \ref{ec:matrix_1} que representa las restricciones de la serie de bucles 
%anidados en el Listado \ref{code:polycode1}, delimitará el espacio de iteraciones descrito 
%mediante un polígono regular. 
%Al tratarse en este caso de un polígono de dos dimensiones se trata de un poliedro, pero 
%en general, para espacios de iteraciones $n$-dimensionales, cuando existen $n$ 
%bucles anidados, se habla del concepto topológico de politopo.


% ---------------------------------------------------
%
% Trabajo de Fin de Grado. 
% Author: Laura Padrón Jorge. 
% Capítulo: La aplicacion BulletPoint. 
% Fichero: Cap4_TheApplication.tex
%
% ----------------------------------------------------
%

\chapter{La aplicación BulletPoint} \label{chap:laaplicacion} 

En este capítulo trataremos diversos temas relacionados con la aplicación, comenzaremos por definir posibles casos de uso en el ámbito universitario, tocaremos diversos temas relacionados incluyendo dificultades durante el desarrollo y acabaremos discutiendo posibles líneas futuras de desarrollo.

 
\section{Aplicaciones móviles en entornos universitarios}


\section {Posibles casos de uso de la tecnología beacon en entornos universitarios}

Actualmente uno de las posibilidades que se presentan para explotar esta tecnología se encuentra en las instituciones de enseñanza, las cuales podría utilizar los beacons para facilitar a sus alumnos, profesores y demás personal involucrado con sus actividades una serie de servicios de gran utilidad.

Sin embargo, para utilizar esta tecnología es necesario cumplir una serie de condiciones:

\begin{itemize}
\item Tener instalada la aplicación en su dispositivo móvil.
\item Tener activado el bluetooth.
\item La aplicación ha de estar despierta.
\item Las beacons han de estar desplegadas y configuradas correctamente en lugares clave donde el rango sea óptimo.
\end{itemize}

%leyenda, en dispositivos Iphone no es necesario tener activado el bluetooth ya que el SO se encarga de captar las señales BLE, aparte tampoco es necesario que la app este despierta que nuevamente el SO se encarga de despertar a la aplicación involucrada. Sin embargo Apple no ha desarrollado una Ibeacon física aún, aunque en un futuro, se espera que sus dispositivos móviles puedan funcionar como una beacon bidirecional.


Asimismo podemos afirmar que prácticamente hoy en día la mayoría de las universidades cuentan con una disposición amplia en los que se refiere a servicios y despliegue de medios. Como ejemplo podemos coger la Universidad de la Laguna, la cual cuenta con una red WiFi con un rango de cobertura casi completo de sus instalaciones y una amplia carta de servicios disponibles a sus alumnos por una serie de medios. Además cuenta con una serie de dispositivos beacons, que podrían ser instalados adecuadamente en lugares estratégicos. 

Partiendo de esta base, procederemos a explorar posibles casos de uso para los beacons en entornos universitarios tomando la ULL como referente:

\subsection{Guía a través del Campus de la ULL}

Este caso de uso cubre la funcionalidad destacada de un beacon, el posicionamiento y guia tanto en exteriores como en interiores. 

Como interesados podríamos destacar: 

\begin{itemize}
\item Personal invitado a jornadas o eventos en instalaciones de la ULL.
\item Alumnos de intercambio en programas internacionales.
\item Alumnos de nuevo acceso.
\item Personas con discapacidad.
\end{itemize}

\subsection{Descarga automática de material}

\subsection{}

\subsection{}

\subsection{}

\subsection{}

\subsection{}

\subsection{}

\section{4 Casos de uso elegidos}
%Arreglar una vez elija el caso de uso 


\section{Despliegue}



% ---------------------------------------------------
%
% Trabajo de Fin de Grado. 
% Author: Laura Padrón Jorge. 
% Capítulo: Conclusiones y lineas de trabajo futuras. 
% Fichero: Cap5_ConclusionsAndFutureLinesOfWork.tex
%
% ----------------------------------------------------
%

\chapter{Conclusiones y lineas de trabajo futuras} \label{chap:conclusiones} 

En este capítulo se presentaran las conclusiones a las que se ha llegado tras realizar este proyecto y discutiremos posibles lineas de trabajo futuras.

\subsection{Conclusiones}


\subsection{Lineas de trabajo futuros}



% ---------------------------------------------------
%
% Trabajo de Fin de Grado. 
% Author: Laura Padrón Jorge. 
% Capítulo: Presupuesto. 
% Fichero: Cap6_BudgetEstimations.tex
%
% ----------------------------------------------------
%

\chapter{Presupuesto y puesta en marcha} \label{chap:presupuesto} 

En este capítulo se expondrá de manera estimada las estimaciones de recursos necesarias para poner en práctica este despliegue, teniendo en cuenta el estado actual del proyecto. 

Actualmente el despliegue de la aplicación se podría separar en dos partes, por un lado tendríamos la compra de los distintos dispositivos y por otro lado hablaríamos del desarrollo y mantenimiento de la aplicación, por lo que se podría contabilizar de la siguiente manera: 

\begin{itemize}
\item Para el caso de uso de los autobuses sería necesario un beacon por parada, si solo se cubriesen las paradas dentro de la zona universitaria, como mucho podríamos estar hablando de unos 20 beacons, distribuidos entre anchieta y guajara.
\item Siguiendo con el caso de uso de los eventos, se podrían calcular unas 10 zonas de eventos principales, nuevamente a distribuir entre guajara y anchieta. Zonas como el paraninfo, el aularium, aulas magnas, bibliotecas o lugares destinados específicamente para la realización de eventos.
\item Los casos de uso que dependen de la localización dependerían de las zonas que habría que controlar, por lo que habría que hacer un análisis de cada recinto en  el caso del parking, aulas para el módulo de asistencia y edificios o exteriores para el módulo de guía.
\end{itemize}


Teniendo en cuenta que cada dispositivo de Aruba cuesta entre 20 a 30 euros.

 
\begin{itemize}
\item Para cubrir el primer caso de uso sería necesaria una inversión de 400 a 600 euros con lo que se cubrirían 20 paradas de autobus.
\item Para las 10 zonas de eventos estaríamos hablando de unos 200 a 300 euros, sería posible añadir más beacons si fuera necesario.
\item Los casos de uso que dependen de la localización son muy relativos, los parkings son cerca de unos 8, teniendo en cuenta que mínimo necesitaríamos unos 3 beacons por recinto, tendríamos un total de 480 a 720 euros, en el caso de la asistencia y guía podríamos calcular para todos los edificios de la ULL unos 100 beacons, con lo que tendríamos cubiertas las facultades en su mayoría con un importe entre 2000 a 3000 eruos.
\end{itemize}


Juntando todo esto nos quedaría un presupuesto de unos 3850 euros de media, y quedarían cubiertas los edificios, parkings y paradas de autobus cercanas al campus. Aparte de los dispositivos habría que añadir el mantenimiento de la aplicación para añadir los datos de los diferentes beacons, incluyendo un servidor para almacenar los datos y realizar las consultas. 



%\newpage{\pagestyle{empty}}
%\thispagestyle{empty}

%\chapter{Presupuesto}
%\label{chapter:Presupuesto}

%\input{cap7.tex}

%%%%%%%%%%%%%%%%%%%%%%%%%%%%%%%%%%%%%%%%%%%%%%%%%%%%%%%%%%%%%%%%%%%%%%%%%%%%%%%

%%%%%%%%%%%%%%%%%%%%%%%%%%%%%%%%%%%%%%%%%%%%%%%%%%%%%%%%%%%%%%%%%%%%%%%%%%%%%%%
%\newpage{\pagestyle{empty}}
%\thispagestyle{empty}
%\begin{appendix}
%
%\chapter{Título del Apéndice 1}
%\label{appendix:1}
%\input{apendice1.tex}
%
%\chapter{Título del Apéndice 2}
%\label{appendix:2}
%\input{apendice2.tex}
%
%\end{appendix}
%%%%%%%%%%%%%%%%%%%%%%%%%%%%%%%%%%%%%%%%%%%%%%%%%%%%%%%%%%%%%%%%%%%%%%%%%%%%%%%
\addcontentsline{toc}{chapter}{Bibliografía}
\bibliographystyle{plain}
\renewcommand{\bibname}{Bibliografía}   %  Para que no aparezca Índice de figuras
\bibliography{bibliografia}

%%%%%%%%%%%%%%%%%%%%%%%%%%%%%%%%%%%%%%%%%%%%%%%%%%%%%%%%%%%%%%%%%%%%%%%%%%%%%%%

\end{document}
